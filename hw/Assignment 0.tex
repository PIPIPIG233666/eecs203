\documentclass[12pt]{exam}

% essential packages
\usepackage{fullpage} % margin formatting
\usepackage{enumitem} % configure enumerate and itemize
\usepackage{amsmath, amsfonts, amssymb, mathtools} % math symbols
\usepackage{xcolor, colortbl} % colors, including in tables
\usepackage{makecell} % thicker \Xhline in table
\usepackage{graphicx} % images, resizing

% sometimes needed packages
\usepackage{wasysym} % Just used for \smiley{}
\usepackage{hyperref} % hyperlinks
\hypersetup{colorlinks=true, urlcolor=blue}
% \usepackage{logicproof} % natural deduction
% \usepackage{tikz} % drawing graphs
% \usetikzlibrary{positioning}
% \usepackage{multicol}
% \usepackage{algpseudocode} % pseudocode

% paragraph formatting
\setlength{\parskip}{6pt}
\setlength{\parindent}{0cm}

% newline after Solution:
\renewcommand{\solutiontitle}{\noindent\textbf{Solution:}\par\noindent}

% less space before itemize/enumerate
\setlist{topsep=0pt}

% creates \filcl to grey out cells for groupwork grading
\newcommand{\filcl}{\cellcolor{gray!25}}

% creates \probnum to get the problem number
\newcounter{probnumcount}
\setcounter{probnumcount}{1}
\newcommand{\probnum}{\arabic{probnumcount}. \addtocounter{probnumcount}{1}}

% use roman numerals by default
\setlist[enumerate]{label={(\roman*)}}

% creates custom list environments for grading guidelines, question parts
\newlist{guidelines}{itemize}{1}
\setlist[guidelines]{label={}, left=0pt .. \parindent, nosep}
\newlist{gwguidelines}{enumerate}{1}
\setlist[gwguidelines]{label={(\roman*)}, nosep}
\newlist{qparts}{enumerate}{2}
\setlist[qparts]{label={(\alph*)}}
\newlist{qsubparts}{enumerate}{2}
\setlist[qsubparts]{label={(\roman*)}}
\newlist{stmts}{enumerate}{1}
\setlist[stmts]{label={(\roman*)}, nosep}
\newlist{pflist}{itemize}{4}
\setlist[pflist]{label={$\bullet$}, nosep}
\newlist{enumpflist}{enumerate}{4}
\setlist[enumpflist]{label={(\arabic*)}, nosep}

\printanswers

\begin{document}
%%%%%%%%%%%%%%% TITLE PAGE %%%%%%%%%%%%%%%
\title{EECS 203: Discrete Mathematics\\
  Winter 2024\\
  Assignment 0: Course Policies \& Page-Matching}
\date{}
\author{}
\maketitle
\vspace{-50pt}
\begin{center}
  \huge Due \textbf{THURSDAY, Jan. 18}, 10:00 pm\\
\Large No late homework accepted past midnight.\\
\vspace{10pt}
\large Number of Problems: 9
\hspace{3cm}
Total Points: 100
\end{center}
\vspace{25pt}
Like all future homeworks, this assignment should be submitted as a PDF through Gradescope (see instructions on course site for more details). We strongly prefer students to compose homework solutions using a word processor (Google Docs or MS Word), or ideally using \LaTeX, but we will accept handwritten homework submissions scanned/photographed and converted to PDF. Note that submitted files must be less than 50 mb in size, but they really should be much smaller than this. No email or Piazza regrade requests will be accepted. For more detail on regrade requests, please refer to course policies.

\begin{itemize}
    \item We will not grade homework problems that were not properly matched on Gradescope. Please submit early and often and double-check that you matched each question to the page(s) where your solution appears.
    \item Explanation or justification for yes/no, true/false, multiple choice questions, and the like: You must provide some sort of explanation or justification for these types of questions (so we know you didn’t just guess), unless explicitly specified in the problem statement. For example, simply answering “true” for a T/F question without providing an explanation will receive little or no credit.
    \item All problems require work to be shown. Questions that require a numerical answer will not be given credit if no work/explanation is shown.
    \item Honor Code: By submitting this homework, you agree that you are in compliance with the Engineering Honor Code and the Course Policies for 203, and that you are submitting your own work.
    \item Hyperlinks on Submitting Homework to Gradescope: \href{https://www.gradescope.com/help#help-center-item-student-submitting}{\textcolor{blue}{Instructions}} or
    \href{https://youtu.be/KMPoby5g_nE}{\textcolor{blue}{Youtube Video}}
\end{itemize}
\newpage
%%%%%%%%%%%%%%% TITLE PAGE %%%%%%%%%%%%%%%

\subsection*{\probnum Did You Read The First Page? [6 points]}
Did you read the first page? We get it, it's a lot of words. Please give it a read! It's important information that you'll want to know as you complete homework assignments throughout the semester to \textbf{avoid losing points from logistical mistakes} \smiley{}. After you've given the first page a careful read over, write what you're most excited about in 2024 as your ``answer" to this question. 
\begin{solution}
Yes, I read the first page. I'm most excited about learning the language of computers because I've always been interested in how computers work.
\end{solution}

\subsection*{\probnum A Thing About Page Matching [6 points]}
You must match your pages to the problems on Gradescope in order to receive credit. Please note as soon as you press submit, you’ve successfully submitted by the deadline. You can still match the pages with no rush, that doesn’t add to your submission time. Write down your favorite class so far to affirm you've read and understand this.
\begin{solution}
I like CLCIV 382 the most because it is a class classified by r/uofm Reddit as a class where you do a "whole lotta nothing" yet I enjoy the material and the professor.
\end{solution}


\subsection*{\probnum Resource Location [12 points]}
Where can you find each of the following resources? No justification necessary.
\begin{qparts}
    \item Lecture slides, lecture recordings, course announcements
    \item Scheduled office hours
    \item Questions and answers about course content and logistics (online forum)
\end{qparts}

\begin{solution}
    Lecture slides, lecture recordings, course announcements: Canvas/files\\Scheduled office hours: Canvas (Google calendar)\\Questions and answers about course content and logistics (online forum): Piazza
\end{solution}


\subsection*{\probnum Exam dates [12 points]}
What are the dates and times of the three course exams? No justification necessary.

\begin{solution}
The three course exams will be from 7-9 pm on February 19, March 27, and April 30.
\end{solution}


\subsection*{\probnum Regrade Requests [16 points]}
\begin{qparts}
    \item Where do you submit a regrade request?
    
    \item What is the deadline for submitting regrade requests? Select \textbf{one} of the following:
    \begin{qsubparts}
        \item by the last day of classes
        \item by the next exam
        \item 24 hours after the assignment's grades have been released
        \item 1 week after the assignment's grades have been released
    \end{qsubparts}
    
    \item When should you submit a regrade request? Select \textbf{any number} of the following:
    \begin{qsubparts}
        \item I feel like I deserve more points
        \item My solution was incorrectly graded according to the posted rubric
        \item I have an alternate solution I think is right
        \item I have an alternate solution I think is right, and I checked it with an instructor in office hours or on Piazza
    \end{qsubparts}
    \item What 3 things should you make sure to (re)read before submitting a regrade request?
\end{qparts}
\begin{solution}

a) A regrade request can be submitted by clicking on the ``Request Regrade`` button on Gradescope.
b) (iv)
c) (ii) and (iv)

\end{solution}

\subsection*{\probnum Admin Form [8 points]}
What is the best way to contact course staff about individual administrative concerns, and where can you find the link? 

\begin{solution}
The admin form is the best way to contact course staff about individual administrative concerns. The link can be found on the canvas homepage.
\end{solution}


\subsection*{\probnum Reflection [8 points]}
What strengths, strategies, and resources do you have to help you succeed in each of following?

\begin{qparts}
    \item Weekly assignments
    \item Exams
\end{qparts}

\begin{solution}
For weekly assignments, I have the strength of being able to work collaboratively with my peers, the strategy of starting early and working consistently, and the resource of office hours. For exams, I have the strength of being able to work collaboratively with my peers, the strategy of starting early, and the resource of office hours and Piazza.\\For exams I have the strength of using ECoach, the strategy of studying for the exams early, and the resource of problem roulette.
\end{solution}


\subsection*{\probnum Exponents [16 points]}
Compute the value of each of the following.  We encourage you to keep it in exponential form as long as you can, for practice. \textbf{Remember to show your work.}
\begin{qparts}
    \item $5^3$
    \item $2^3\cdot 2^2$
    \item $\frac{3^6}{3^3}$
    \item $5^0$
    \item $(2^4)^2$
    \item $2^{(4^2)}$ (use a calculator for this one if you need to!)
    \item $16^{\frac 12}$
    \item $16^{-2}$
\end{qparts}

\begin{solution}

\end{solution}

\subsection*{\probnum Logarithms [16 points]}
Compute the value of each of the following.  If the answer does not come out cleanly and results in an infinite decimal expansion, write ``\textit{too hard to compute}." If the result is not defined, write ``\textit{undefined}." For any question referencing $x$ or $y,$ let $\log_5(x)=2.6$ and $\log_5(y)=5.2.$ \textbf{Remember to show your work.}
\begin{qparts}
    \item $\log_2{64}$
    \item $\log_7{7}$
    \item $\log_7{1}$
    \item $\log_7{0}$
    \item $\log_5(x+y)$
    \item $\log_5(xy)$
    \item $\log_2(x)$
    \item $\log_5(x^{25})$
\end{qparts}

\begin{solution}

\end{solution}

\end{document}
