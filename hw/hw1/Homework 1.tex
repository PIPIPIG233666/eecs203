\documentclass[12pt]{exam}

% essential packages
\usepackage{fullpage} % margin formatting
\usepackage{enumitem} % configure enumerate and itemize
\usepackage{amsmath, amsfonts, amssymb, mathtools} % math symbols
\usepackage{xcolor, colortbl} % colors, including in tables
\usepackage{makecell} % thicker \Xhline in table
\usepackage{graphicx} % images, resizing

% sometimes needed packages
\usepackage{hyperref} % hyperlinks
% \hypersetup{colorlinks=true, urlcolor=blue}
% \usepackage{logicproof} % natural deduction
% \usepackage{tikz} % drawing graphs
% \usetikzlibrary{positioning}
% \usepackage{multicol}
% \usepackage{algpseudocode} % pseudocode

% paragraph formatting
\setlength{\parskip}{6pt}
\setlength{\parindent}{0cm}

% newline after Solution:
\renewcommand{\solutiontitle}{\noindent\textbf{Solution:}\par\noindent}

% less space before itemize/enumerate
\setlist{topsep=0pt}

% creates \filcl to grey out cells for groupwork grading
\newcommand{\filcl}{\cellcolor{gray!25}}

% creates \probnum to get the problem number
\newcounter{probnumcount}
\setcounter{probnumcount}{1}
\newcommand{\probnum}{\arabic{probnumcount}. \addtocounter{probnumcount}{1}}

% use roman numerals by default
\setlist[enumerate]{label={(\roman*)}}

% creates custom list environments for grading guidelines, question parts
\newlist{guidelines}{itemize}{1}
\setlist[guidelines]{label={}, left=0pt .. \parindent, nosep}
\newlist{gwguidelines}{enumerate}{1}
\setlist[gwguidelines]{label={(\roman*)}, nosep}
\newlist{qparts}{enumerate}{2}
\setlist[qparts]{label={(\alph*)}}
\newlist{qsubparts}{enumerate}{2}
\setlist[qsubparts]{label={(\roman*)}}
\newlist{stmts}{enumerate}{1}
\setlist[stmts]{label={(\roman*)}, nosep}
\newlist{pflist}{itemize}{4}
\setlist[pflist]{label={$\bullet$}, nosep}
\newlist{enumpflist}{enumerate}{4}
\setlist[enumpflist]{label={(\arabic*)}, nosep}

\printanswers

\newcommand{\prevhwnum}{0}
\newcommand{\hwnum}{1}

\begin{document}
%%%%%%%%%%%%%%% TITLE PAGE %%%%%%%%%%%%%%%
\title{EECS 203: Discrete Mathematics\\
  Winter 2024\\
  Homework \hwnum{}}
\date{}
\author{}
\maketitle
\vspace{-50pt}
\begin{center}
  \huge Due \textbf{Thursday, January 25th}, 10:00 pm\\
\Large No late homework accepted past midnight.\\
\vspace{10pt}
\large Number of Problems: $7+1$
\hspace{3cm}
Total Points: $100+20$
\end{center}
\vspace{25pt}
\begin{itemize}
    \item \textbf{Match your pages!} Your submission time is when you upload the file, so the time you take to match pages doesn't count against you.
    \item Submit this assignment (and any regrade requests later) on Gradescope. 
    \item Justify your answers and show your work (unless a question says otherwise).
    \item By submitting this homework, you agree that you are in compliance with the Engineering Honor Code and the Course Policies for 203, and that you are submitting your own work.
    \item Check the syllabus for full details.
\end{itemize}
\newpage
%%%%%%%%%%%%%%% TITLE PAGE %%%%%%%%%%%%%%% 

\section*{Individual Portion}

\subsection*{\probnum Collaboration and Support [3 points]}
\begin{qparts}
    \item Give the names and uniqnames of 2 of your EECS 203 classmates (these could be members of your homework group or other classmates). 
    \item When you have questions about the course content, where can you ask them? Where are \textit{you} most likely to ask questions?
    \item Name one self-care action you plan to do this semester to maintain your overall well-being. 
\end{qparts}

\begin{solution}
  a) Youssef Cherri (ycherri), Johnathan Kertawidjaja (jonkert)\\
  b) We go to the same discussion section, so I would ask them there. I would most likely ask questions where they are hard.\\
  c) I take naps when I can't focus anymore.

\end{solution}


\subsection*{\probnum Rock the Vote [12 points]}
Let $p$ and $q$ be the following propositions:
\begin{itemize}
    \item $p$: The election has been decided.
    \item $q$: The votes have been counted.
\end{itemize}
Express each of these propositions as an English statement:
\begin{qparts}
    \item $\neg p\rightarrow\neg q$
    \item $\neg q \lor (\neg p \land q)$
\end{qparts}

\begin{solution}
a) If the election has not been decided, then the votes have not been counted.\\
b) The votes have not been counted, or the election has not been decided and the votes have been counted.
\end{solution}


\subsection*{\probnum Negation Station [16 points]}

For each of the following propositions, give their negation in natural English. Your answer should not contain the original proposition. That is, you shouldn't negate it as ``It is not the case that ...'' or something similar.

\textbf{Note:} You do not need to show work besides your translation, but you may earn partial credit if you do.

\begin{qparts}
    \item $a$ is greater than 6 or at most 2.
    \item $b$ is a perfect square, odd, and not divisible by 7.
    \item $c$ is odd whenever it is prime and greater than 3.
    \item If $d$ is divisible by 2, then it is even.
\end{qparts}

\begin{solution}
a) a is less than 6 and at least 2.
\\b) b is not a perfect square, is even, and is divisible by 7
\\c) c is even whenever it is not prime and is smaller than 3
\\d) d is not divisible by 2 and is odd ??
\end{solution}


\subsection*{\probnum Lying and Politics [16 points]}
Imagine a world with two kinds of people: knights and knaves, where knights always tell the truth and knaves always lie. There are three people A, B, and C, and one of them is the city mayor.
\begin{itemize}
    \item A says ``I am not the city mayor."
    \item B says ``The city mayor is a knave."
    \item C says ``All three of us are knaves."
\end{itemize}
Is the city mayor a knight or a knave? As part of your solution, determine everything you can about whether A, B, and C are knights or knaves.

\begin{solution}

\end{solution}


\subsection*{\probnum Is Equivalence Equivalent to Equality? [15 points]}
Show that $(b \rightarrow a) \land (c \rightarrow a)$ is logically equivalent to $\neg(b \lor c) \lor a$.  If you use a truth table, be sure to state why the table tells you what you claim.  If you use logical equivalences, be sure to cite each law you use.

\begin{solution}
$(b \rightarrow a) \land (c \rightarrow a)$ is logically equivalent to $\neg(b \lor c) \lor a$.
\\ $\neg(b \lor c) \land a $ = $\neg(b \lor a) \land \neg(c \lor a)$ (Distributibe Laws).
\\ = $\neg(b) \land \neg(c) \land a$ (DeMorgan's Laws).
\\ $(b \rightarrow a) \land (c \rightarrow a)$ = $(\neg b \lor a) \land (\neg c \lor a)$ (implication breakout rule).
\end{solution}

\subsection*{\probnum Deduce, Reuse, Recycle [20 points]}

Given that the following statements are \textbf{true}:
\[ (p \land r) \rightarrow q\hspace{0.5in}\: \neg q  \hspace{0.5in} \: r \lor s \hspace{0.5in} \: q \lor r \]
For each of the propositions, $p,\ q,\ r,$ and $s$, state its truth value and explain. If it cannot be found, briefly explain why. 
\begin{solution}

\end{solution}

\subsection*{\probnum Preposterous Propositions [18 points]}

Consider the following truth table, where $s$, $t$, and $w$ are unknown propositions.

\begin{center}
\begin{tabular}{|c c c|| c | c | c |}
\hline
$p$ & $q$ & $r$ & 
\hspace{0.8cm}$s$\hspace{0.8cm} & \hspace{0.8cm}$t$\hspace{0.8cm} & \hspace{0.8cm}$w$\hspace*{0.8cm}\\\hline
T & T & T  & F & T & F\\\hline
T & T & F  & T & F & F\\\hline
T & F & T  & F & T & T\\\hline
T & F & F  & F & T & F\\\hline
F & T & T  & F & T & T\\\hline
F & T & F  & F & F & F\\\hline
F & F & T  & F & T & T\\\hline
F & F & F  & F & T & F\\\hline
\end{tabular}
\end{center}

Use the above truth table to answer the following questions. For each unknown proposition, $s$, $t$, and $w$: 
\begin{itemize}
    \item Find an equivalent compound proposition using $p$, $q$, and/or $r$. 
    \item You may use \textbf{only} $\land$, $\lor$, $\neg$, and parentheses in each of your answers.  
    \item You may use $p$, $q$, and $r$ \textbf{at most once} in each of your answers.
\end{itemize}

\begin{solution}

\end{solution}


\pagebreak
\setcounter{probnumcount}{1}
\section*{Groupwork \hwnum{} Problems}
\subsection*{\probnum Caught Red-Handed! [20 points]}
Four friends have been identified as suspects for a recent hack. They have made the follow statements to the authorities:
\begin{itemize}
    \item Redd says that ``Blu did it"
    \item Violet says that ``I did not do it"
    \item Blu says that ``Grey did it"
    \item Grey says that ``Blu lied when they said that I did it"
\end{itemize}
\begin{qparts}
    \item If the authorities know exactly one of the four suspects is telling the truth, who did it?
    \item If the authorities know exactly one of the four suspects is lying, who did it?
\end{qparts}

\begin{solution}

\end{solution}


\end{document}
