\documentclass[12pt]{exam}

% essential packages
\usepackage{fullpage} % margin formatting
\usepackage{enumitem} % configure enumerate and itemize
\usepackage{amsmath, amsfonts, amssymb, mathtools} % math symbols
\usepackage{xcolor, colortbl} % colors, including in tables
\usepackage{makecell} % thicker \Xhline in table
\usepackage{graphicx} % images, resizing

% sometimes needed packages
\usepackage{hyperref} % hyperlinks
% \hypersetup{colorlinks=true, urlcolor=blue}
\usepackage{logicproof} % natural deduction
% \usepackage{tikz} % drawing graphs
% \usetikzlibrary{positioning}
% \usepackage{multicol}
% \usepackage{algpseudocode} % pseudocode

% paragraph formatting
\setlength{\parskip}{6pt}
\setlength{\parindent}{0cm}

% newline after Solution:
\renewcommand{\solutiontitle}{\noindent\textbf{Solution:}\par\noindent}

% less space before itemize/enumerate
\setlist{topsep=0pt}

% creates \filcl to grey out cells for groupwork grading
\newcommand{\filcl}{\cellcolor{gray!25}}
\newcommand{\divides}{\,|\,}

% creates \probnum to get the problem number
\newcounter{probnumcount}
\setcounter{probnumcount}{1}
\newcommand{\probnum}{\arabic{probnumcount}. \addtocounter{probnumcount}{1}}

% use roman numerals by default
\setlist[enumerate]{label={(\roman*)}}

% creates custom list environments for grading guidelines, question parts
\newlist{guidelines}{itemize}{1}
\setlist[guidelines]{label={}, left=0pt .. \parindent, nosep}
\newlist{gwguidelines}{enumerate}{1}
\setlist[gwguidelines]{label={(\roman*)}, nosep}
\newlist{qparts}{enumerate}{2}
\setlist[qparts]{label={(\alph*)}}
\newlist{qsubparts}{enumerate}{2}
\setlist[qsubparts]{label={(\roman*)}}
\newlist{stmts}{enumerate}{1}
\setlist[stmts]{label={(\roman*)}, nosep}
\newlist{pflist}{itemize}{4}
\setlist[pflist]{label={$\bullet$}, nosep}
\newlist{enumpflist}{enumerate}{4}
\setlist[enumpflist]{label={(\arabic*)}, nosep}

\printanswers

\newcommand{\prevhwnum}{1}
\newcommand{\hwnum}{2}

\begin{document}
%%%%%%%%%%%%%%% TITLE PAGE %%%%%%%%%%%%%%%
\title{EECS 203: Discrete Mathematics\\
	Winter 2024\\
	Homework \hwnum{}}
\date{}
\author{}
\maketitle
\vspace{-50pt}
\begin{center}
	\huge Due \textbf{Thursday, Feb. 1st}, 10:00 pm\\
	\Large No late homework accepted past midnight.\\
	\vspace{10pt}
	\large Number of Problems: $8+2$
	\hspace{3cm}
	Total Points: $100+40$
\end{center}
\vspace{25pt}
\begin{itemize}
	\item \textbf{Match your pages!} Your submission time is when you upload the file, so the time you take to match pages doesn't count against you.
	\item Submit this assignment (and any regrade requests later) on Gradescope.
	\item Justify your answers and show your work (unless a question says otherwise).
	\item By submitting this homework, you agree that you are in compliance with the Engineering Honor Code and the Course Policies for 203, and that you are submitting your own work.
	\item Check the syllabus for full details.
\end{itemize}
\newpage
\pagebreak
\section*{Grading of Groupwork 2}
Using the solutions and Grading Guidelines, grade your Groupwork 2 Problems:
\begin{itemize}
    \item Use the table below to grade your past groupwork submission and calculate scores.
    \item While grading, mark up your past submission. Include this with the table when you submit your grading.
    \item Write whether your submission achieved each rubric item. If it didn't achieve one, say why not.
    \item For extra credit, write positive comment(s) about your work.
    \item You don't have to redo problems correctly, but it is recommended!
    \item See ``All About Groupwork" on Canvas for more detailed guidance, and what to do if you change groups.
\end{itemize}

\begin{center}
\resizebox{\textwidth}{!}{\begin{tabular}{| c | c | c | c | c | c | c | c | c | c | c | c | c |}
\hline
 & (i) & (ii) & (iii) & (iv) & (v) & (vi) & (vii) & (viii) & (ix) & (x) & (xi) & Total:\\
\hline
Problem 1 & +4 & +4  & +2 & +2& +2& +4& +2&\filcl &\filcl & \filcl& \filcl& \hspace{1cm}/20\\
\hline
Problem 2 & +5&+0 &+5 &+0 &\filcl &\filcl &\filcl &\filcl &\filcl & \filcl& \filcl& \hspace{1cm}/20\\
\Xhline{1.25pt}
Total: &\filcl &\filcl &\filcl &\filcl &\filcl &\filcl &\filcl &\filcl & \filcl& \filcl& \filcl&\hspace{1cm}/40\\
\hline
\end{tabular}}
\end{center}

\pagebreak

\pagebreak
\setcounter{probnumcount}{1}
\section*{Groupwork \hwnum{} Problems}

\subsection*{\probnum Bézout's Identity [20 points]}

In number theory, there's a simple yet powerful theorem called Bézout's identity, which states that for any two integers $a$ and $b$ (with $a$ and $b$ not both zero) there exist two integers $r$ and $s$ such that $ar+bs=\gcd(a,b).$ Use Bézout's identity to prove the following statements (you may assume all variables are integers):

\begin{qparts}
	\item If $d\divides a$ and $d\divides b,$ then $d\divides \gcd(a,b).$
	\item If $a\divides bc$ and $\gcd(a,b)=1,$ then $a\divides c.$
\end{qparts}

\noindent
\textbf{Note:} $\gcd$ is short for ``greatest common divisor," so the value of $\gcd(a,b)$ is the largest integer that evenly divides $a$ and $b.$ You won't need to apply this definition, just know that $\gcd(a,b)$ is an integer.


\begin{solution}
\begin{tabular}{ll}
  a) Assume $d \mid a \land d \mid b$ & premise\\
  $a=kd \land b=jd$, k and j are arbitary integers &definition of divide\\
  $kdr+jds=gcd(a,b)$ &Bezout's Identity\\
  $gcd(a,b)=d(kr+js)$ &factoring \\
  Since $kr+js$ is also an integer, thus $d \mid gcd(a,b)$ &definition of divide\\
  Thus, $d \mid gcd(a,b)$.\\
  \\b) Assume $d \mid bc \land gcd(a,b)=1$ &premise
  \\$bc=ka$ &definition of divide
  \\$ar+bs=1$ &Bezout's Identity\\
  Since c is an non-zero integer, $arc+bsc=c$ &equation\\
  $arc+kac=c$ &subsitution\\
  $a(rc+kc)=c$ &factoring\\
  Since $rc+kc$ is also an integer, thus $a\mid c$ &definition of divide\\
  Thus, $a\mid c$\\

\end{tabular}
\end{solution}


\subsection*{\probnum Proposition Michigan [20 points]}
Translate each of the following English statements into logic. You may define predicates as necessary.

\textbf{Note:} Your predicates should not trivialize the problem.

\begin{qparts}
	\item Each pair of students at UMich has at least two mutual friends at UMich. The domain of discourse is all students at UMich.

	\item Nobody knows everyone's Wolverine Access password except the Wolverine Access administrators, who know all passwords. The domain of discourse is all people who have a Wolverine Access account (the administrators have Wolverine Access accounts).
\end{qparts}

\begin{solution}
 a) Let F(x) be UMich students who are friends
 \\$\forall x \forall y [x \neq y \rightarrow \exists z \exists w[z \neq w \land F(x)] ]$\\
\textcolor{red} {I have a different predicate defined that should also work}
\\b) Let A(x) be is administrator, P(x,\textcolor{red}{y}) be y knows someone's password\\
$\forall x(A(x)\iff \textcolor{red} \forall y P(x,y))$
\\Comments:\\
I did fine after camping at OH for 3 hours straight.
\end{solution}

\end{document}
