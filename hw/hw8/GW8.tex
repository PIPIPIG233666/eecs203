\documentclass[12pt]{exam}

% essential packages
\usepackage{fullpage} % margin formatting
\usepackage{enumitem} % configure enumerate and itemize
\usepackage{amsmath, amsfonts, amssymb, mathtools} % math symbols
\usepackage{xcolor, colortbl} % colors, including in tables
\usepackage{makecell} % thicker \Xhline in table
\usepackage{graphicx} % images, resizing

% sometimes needed packages
\usepackage{hyperref} % hyperlinks
\usepackage{tikz} % drawing graphs
\usetikzlibrary{positioning}

% paragraph formatting
\setlength{\parskip}{6pt}
\setlength{\parindent}{0cm}

% newline after Solution:
\renewcommand{\solutiontitle}{\noindent\textbf{Solution:}\par\noindent}

% less space before itemize/enumerate
\setlist{topsep=0pt}

% creates \filcl to grey out cells for groupwork grading
\newcommand{\filcl}{\cellcolor{gray!25}}

% creates \probnum to get the problem number
\newcounter{probnumcount}
\setcounter{probnumcount}{1}
\newcommand{\probnum}{\arabic{probnumcount}. \addtocounter{probnumcount}{1}}

% use roman numerals by default
\setlist[enumerate]{label={(\roman*)}}

% creates custom list environments for grading guidelines, question parts
\newlist{guidelines}{itemize}{1}
\setlist[guidelines]{label={}, left=0pt .. \parindent, nosep}
\newlist{gwguidelines}{enumerate}{1}
\setlist[gwguidelines]{label={(\roman*)}, nosep}
\newlist{qparts}{enumerate}{2}
\setlist[qparts]{label={(\alph*)}}
\newlist{qsubparts}{enumerate}{2}
\setlist[qsubparts]{label={(\roman*)}}
\newlist{stmts}{enumerate}{1}
\setlist[stmts]{label={(\roman*)}, nosep}
\newlist{pflist}{itemize}{4}
\setlist[pflist]{label={$\bullet$}, nosep}
\newlist{enumpflist}{enumerate}{4}
\setlist[enumpflist]{label={(\arabic*)}, nosep}

\printanswers

\newcommand{\prevhwnum}{7}
\newcommand{\hwnum}{8}

\begin{document}
\pagebreak
\setcounter{probnumcount}{1}
\section*{Groupwork \hwnum{} Problems}

\subsection*{\probnum Commit Tea Party [15 points]}
Two committees are having a meeting. If there are 12 people in each committee, how many different ways can they sit around a table given the following restrictions? Note that two orderings are considered equal if each person has the same two neighbors (without distinguishing their left and right neighbors).

\begin{qparts}
	\item There are no restrictions on seating.
	\item Two people in the same committee cannot be neighbors.
	\item Everybody must have exactly two neighbors from their committee.
	\item Everybody must have exactly one neighbor from their committee.
\end{qparts}

\begin{solution}
There are $24$ people in total and $2$ committees.\\
a) No restrictions on seating.\\
Therefore, it is a combination problem.\\
$C(24, 12) = \frac{24!}{12!12!} = 2704156$ ways.\\
b) Two people in the same committee cannot be neighbors.\\
Therefore, it is a permutation problem.\\
$P(24,2) = 24\cdot 23 = 552$ ways.\\
c) Everybody must have exactly two neighbors from their committee.\\
Therefore, it is a permutation problem.\\
$P(24,3) = 24\cdot 23\cdot 22 = 12144$ ways.\\
d) Everybody must have exactly one neighbor from their committee.\\
Therefore, it is a permutation problem.\\
$P(24,2) = 24\cdot 23 = 552$ ways.\\
\end{solution}

\subsection*{\probnum Hiking Extravaganza [15 points]}
Prove that every complete $n$-node weighted graph (with all possible edges) with $n \ge 1$ and all distinct edge weights has a (possibly non-simple) path of $n-1$ edges along which the edge weights are strictly increasing.

\textbf{Hint:} Start by placing a hiker on each node. Try to show that the hikers can walk paths of \emph{total} length $n(n-1)$, each along increasing-weight paths.

\begin{solution}
Let $G$ be a complete $n$-node weighted graph with all distinct edge weights.\\
Let $v_1, v_2, \ldots, v_n$ be the nodes of $G$.\\
Let $w_{ij}$ be the weight of the edge between $v_i$ and $v_j$.\\
Let $H$ be a complete $n$-node weighted graph with all distinct edge weights.\\
Let $u_1, u_2, \ldots, u_n$ be the nodes of $H$.\\
Let $x_{ij}$ be the weight of the edge between $u_i$ and $u_j$.\\
Let $h_{ij}$ be the weight of the edge between $v_i$ and $u_j$.\\
Let $h_{ij} = w_{ij} + \sum_{k=1}^{j-1} x_{k(k+1)}$ for $1 \le i \le n$ and $1 \le j \le n$.\\

\end{solution}

\end{document}

