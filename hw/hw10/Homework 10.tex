\documentclass[12pt]{exam}

% essential packages
\usepackage{fullpage} % margin formatting
\usepackage{enumitem} % configure enumerate and itemize
\usepackage{amsmath, amsfonts, amssymb, mathtools} % math symbols
\usepackage{xcolor, colortbl} % colors, including in tables
\usepackage{makecell} % thicker \Xhline in table
\usepackage{graphicx} % images, resizing

% sometimes needed packages
\usepackage{hyperref} % hyperlinks
% \hypersetup{colorlinks=true, urlcolor=blue}
% \usepackage{logicproof} % natural deduction
\usepackage{tikz} % drawing graphs
\usetikzlibrary{positioning}
\usepackage{multicol}
% \usepackage{algpseudocode} % pseudocode

% paragraph formatting
\setlength{\parskip}{6pt}
\setlength{\parindent}{0cm}

% newline after Solution:
\renewcommand{\solutiontitle}{\noindent\textbf{Solution:}\par\noindent}

% less space before itemize/enumerate
\setlist{topsep=0pt}

% creates \filcl to grey out cells for groupwork grading
\newcommand{\filcl}{\cellcolor{gray!25}}

% creates \probnum to get the problem number
\newcounter{probnumcount}
\setcounter{probnumcount}{1}
\newcommand{\probnum}{\arabic{probnumcount}. \addtocounter{probnumcount}{1}}

% use roman numerals by default
\setlist[enumerate]{label={(\roman*)}}

% creates custom list environments for grading guidelines, question parts
\newlist{guidelines}{itemize}{1}
\setlist[guidelines]{label={}, left=0pt .. \parindent, nosep}
\newlist{gwguidelines}{enumerate}{1}
\setlist[gwguidelines]{label={(\roman*)}, nosep}
\newlist{qparts}{enumerate}{2}
\setlist[qparts]{label={(\alph*)}}
\newlist{qsubparts}{enumerate}{2}
\setlist[qsubparts]{label={(\roman*)}}
\newlist{stmts}{enumerate}{1}
\setlist[stmts]{label={(\roman*)}, nosep}
\newlist{pflist}{itemize}{4}
\setlist[pflist]{label={$\bullet$}, nosep}
\newlist{enumpflist}{enumerate}{4}
\setlist[enumpflist]{label={(\arabic*)}, nosep}

\printanswers

\newcommand{\prevhwnum}{9}
\newcommand{\hwnum}{10}

\begin{document}
%%%%%%%%%%%%%%% TITLE PAGE %%%%%%%%%%%%%%%
\section*{Individual Portion}

\subsection*{\probnum Color Conundrum [14 points]}
Each day Donovan Edwards either wears a T-shirt or a tank top. On a given day, there is a $70\%$ chance he wears a T-shirt and a $30\%$ chance he wears a tank top. If he wears a T-shirt, he randomly picks one of 4 pink T-shirts, 3 blue T-shirts, and 2 black T-shirts (he is equally likely to pick any particular shirt). If he wears a tank top, he randomly picks one of 2 pink tank tops, 3 white tank tops, or 2 blue tank tops.

\begin{qparts}
	\item What is the probability that he is wearing pink or white on a given day?
	\item Given that Donovan is wearing pink or white on a given day, what is the probability that he is wearing a T-shirt?
\end{qparts}

You \textbf{do not} need to simplify your answers.

\begin{solution}
	% separate into two cases, T and TT
	Let $T$ be the event that Donovan wears a T-shirt and $TT$ be the event that Donovan wears a tank top.
	Let $P$, $W$, and $B$ be the events that Donovan wears pink, white, and black, respectively.
	\begin{qparts}
		\item $P(T) = \frac{7}{10}, P(TT) =\frac{3}{10}$ \\
		$P(P | T) = \frac{4}{9}, P(W | T) = \frac{3}{9}, P(B | T) = \frac{2}{9}$ \\
		% dont use T^c, use TT
		$P(P | TT) = \frac{2}{7}, P(W | TT) = \frac{3}{7}, P(B | TT) = \frac{2}{7}$ \\
		% dont simplify
		% basically P(P or W) = P(P) + P(W), 7/10*4/9 + 3/10*(2/7 + 3/7)
		$P(P \cup W) = P(P) + P(W) = \frac{7}{10} \cdot \frac{4}{9} + \frac{3}{10} \cdot \left(\frac{2}{7} + \frac{3}{7}\right)$
		% use conditional probability formula
		\item $P(T | P \cup W) = \frac{P(T \cap (P \cup W))}{P(P \cup W)} = \frac{P(T \cap P) + P(T \cap W)}{P(P) + P(W)}$
		% dont simplify
		$= \frac{\frac{7}{10} \cdot \frac{4}{9} + 0}{\frac{7}{10} \cdot \frac{4}{9} + \frac{3}{10} \cdot \left(\frac{2}{7} + \frac{3}{7}\right)}$

	\end{qparts}

\end{solution}

\subsection*{\probnum Bayes' $\times 3$ [8 points]}
Suppose that $E$, $F_1$, $F_2$, and $F_3$ are events from a sample space $S.$ Furthermore, suppose that $F_1$, $F_2$, and $F_3$ are each mutually exclusive, and that their union is $S$. Find $P(F_2 \mid E)$ if
\begin{align*}
	P(E \mid F_2) & = \frac 38 & P(F_1) & = \frac 16 \\
	P(E \mid F_3) & = \frac 12 & P(F_2) & = \frac 12 \\
	P(E \mid F_1) & = \frac 27 & P(F_3) & = \frac 13
\end{align*}
Express your final answer as a \textbf{single, fully-simplified} number.

\begin{solution}
	% bayes's theorem
	By Bayes's Theorem,
	$P(F_2 | E) = \frac{P(E | F_2) \cdot P(F_2)}{P(E)} = \frac{P(E | F_2) \cdot P(F_2)}{P(E | F_1) \cdot P(F_1) + P(E | F_2) \cdot P(F_2) + P(E | F_3) \cdot P(F_3)}$
	$= \frac{\frac{3}{8} \cdot \frac{1}{2}}{\frac{2}{7} \cdot \frac{1}{6} + \frac{3}{8} \cdot \frac{1}{2} + \frac{1}{2} \cdot \frac{1}{3}}$
	% = 7/15
	$= \frac{7}{15}$
\end{solution}

\subsection*{\probnum There Snow Way I'm Running In This [12 points] }
Ishaan likes to run, but he hates running in the snow. If it snows, the probability of Ishaan going for a run is $\frac 1{10}$. If it doesn't snow, the probability of Ishaan running is $\frac 8{10}$. If Ishaan goes for run, then the probability that it snowed is $\frac 19.$ What is the probability that it snows?

Express your final answer as a \textbf{single, fully-simplified} number.

\begin{solution}
	% the answer is 1/2
	Let $S$ be the event that it snows and $R$ be the event that Ishaan goes for a run.\\
	$P(R | S) = \frac{1}{10}, P(R | S^c) = \frac{8}{10}, P(S | R) = \frac{1}{9}$\\
	By Bayes's Theorem, $P(S | R) = \frac{P(R | S) \cdot P(S)}{P(R)} = \frac{P(R | S) \cdot P(S)}{P(R | S) \cdot P(S) + P(R | S^c) \cdot P(S^c)}$\\
	$= \frac{\frac{1}{10} \cdot P(S)}{\frac{1}{10} \cdot P(S) + \frac{8}{10} \cdot (1 - P(S))}$\\
	$\frac{1}{9} = \frac{P(S)}{P(S) + 8(1 - P(S))}$\\
	$\frac{1}{9} = \frac{P(S)}{9 - 7P(S)}$\\
	$9P(S) = 8 - 7P(S)$\\
	$16P(S) = 8$\\
	P(S) = $\frac{1}{2}$
	\\Therefore, the probability that it snows is $\frac{1}{2}$.
\end{solution}

\subsection*{\probnum What did you expect? [12 points]}

The EECS 203 staff is going on a road trip! The 36 staff members have decided to split up into 6 different cars with 9, 8, 6, 6, 4, 3 people in each of the respective cars.
\begin{qparts}
	\item Suppose we pick a car uniformly at random, and consider $X$ to be the random variable defined by the number of staff members in that car. What is the expected value of $X$?
	\item Now suppose we pick one of the staff members uniformly at random. Let $Y$ be the random variable defined by the number of people in the car that staff member is in. What is the expected value of $Y$?
\end{qparts}
Express your final answers as \textbf{single, fully-simplified} numbers.

\begin{solution}
	\begin{qparts}
		\item
		% use linearity of expectation
		It is equally likely to pick any car.
		$E[X] = 9 \cdot \frac{1}{6} + 8 \cdot \frac{1}{6} + 6 \cdot \frac{1}{6} + 6 \cdot \frac{1}{6} + 4 \cdot \frac{1}{6} + 3 \cdot \frac{1}{6}$
		$= \frac{9 + 8 + 6 + 6 + 4 + 3}{6} = 6$
		\item
		% use law of total expectation
		The probability of getting picked depends on the car they are in.
		i.e. more likely to be picked if in a car with more people.
		$E[Y] = 9 \cdot \frac{9}{36} + 8 \cdot \frac{8}{36} + 6 \cdot \frac{6}{36} + 6 \cdot \frac{6}{36} + 4 \cdot \frac{4}{36} + 3 \cdot \frac{3}{36}$
		$= \frac{9^2 + 8^2 + 6^2 + 6^2 + 4^2 + 3^2}{36}$
		% 121/18
		$= \frac{121}{18}$
	\end{qparts}

\end{solution}

\subsection*{\probnum Zero-sum game...or is it? [12 points]}

Your friend proposes to play the following game. You roll a fair, 6-sided dice twice and record the result. Let $X$ be the random variable defined as twice the value of the first roll, minus three times the value of the second roll. For example, if you rolled $3$ then $4,$ then $X$ would equal $2\cdot 3 - 3\cdot 4=-6.$ You win $X$ dollars if $X$ is positive, but have to give your friend $|X|$ dollars if $X$ is negative. If $X$ is zero then you neither win nor lose money. How much money do you expect to win or lose?

Express your final answer as a \textbf{single, fully-simplified} number.

\begin{solution}
	% use linearity of expectation
	Let $Y$ be the random variable for the value of the first roll and $Z$ be the random variable for the value of the second roll.\\
	Since the die is fair, the probability of each number is $\frac{1}{6}$.\\
	$E[Y] = n \cdot P(Y = n) = \frac{1+2+3+4+5+6}{6} = \frac{21}{6} = 3.5$\\
	$E[Z] = n \cdot P(Z = n) = \frac{1+2+3+4+5+6}{6} = \frac{21}{6} = 3.5$\\
	$E[X] = 2E[Y] - 3E[Z] = 2 \cdot 3.5 - 3 \cdot 3.5 = -3.5$\\
	Therefore, you expect to lose \$3.50.

\end{solution}

\subsection*{\probnum Rollie Pollie [15 points]}

Rohit recently became super passionate about rolling dice. He decides to roll a single fair 6-sided die 100 times. What is the expected number of times he rolls a 5 followed by a 6?

Express your final answer as a \textbf{single, fully-simplified} number.

\begin{solution}
	Let $I_k$ be the indicator random variable for the event that the $k$th roll is a 5 and the $(k+1)$th roll is a 6.\\
	$X$ be the random variable for the number of times Rohit rolls a 5 followed by a 6.
	% E(I_k) = 1 * P(I_k = 1) + 0 * P(I_k = 0)
	$E[I_k] = 1 \cdot P(I_k = 1) + 0 \cdot P(I_k = 0)$
	% P(I_k = 1) = 1/36
	\\Only 1 time out of $6^2$ rows will Rohit get a 5 followed by a 6, thus $P(I_k = 1) = \frac{1}{36}$
	% E(I_k) = 1/36
	$E[I_k] = \frac{1}{36}$\\
	\\By linearity of expectation,
	$X = I_1 + I_2 + \dots + I_{99}$
	\\$E[X] = E[I_1] + E[I_2] + \dots + E[I_{99}]$
		% E(X) = 99/36
	$E[X] = \frac{99}{36}$
		\\Therefore, the expected number of times Rohit rolls a 5 followed by a 6 is $\frac{11}{4}$.

\end{solution}

\subsection* {\probnum Bernoulli trials, binomial distribution [15 points]}

You roll a fair six-sided die 12 times. Find:
\begin{qparts}
	\item The probability that exactly two rolls come up as a 6.
	\item The probability that exactly two rolls come up as a 6, given that the first four rolls each came up as 3.
	\item The probability that at least two rolls come up as a 6.
	\item The expected number of rolls that come up as 6.
\end{qparts}

You \textbf{do not} need to simplify your answers.

\begin{solution}
	This problem satisfies the conditions for a binomial distribution because there are a fixed number of trials, each trial is independent, and each trial has the same probability of success.\\
	Let $X$ be the random variable for the number of times a 6 is rolled.\\
	$I_k$ be the indicator random variable for the event that the $k$th roll is a 6.\\
	Since the die is fair, $E[I_k] = \frac{1}{6}$.
	\begin{qparts}
		\item By the binomial distribution and Bernoulli Trials, the success is getting a 6, so $p = \frac{1}{6}$. The probability that exactly 2 successes in 12 trials is $\binom{12}{2} \left(\frac{1}{6}\right)^2 \left(1-\frac{1}{6}\right)^{10}$.
		\item Similarly, the probability that exactly two rolls come up as a 6, given that the first four rolls each came up as 3 is $\binom{8}{2} \left(\frac{1}{6}\right)^2 \left(1-\frac{1}{6}\right)^{6}$. This is because the two events are independent. The first four rolls do not affect the probability of the next 8 rolls.
		\item The probability that at least two rolls come up as a 6 is 1 mius the probability that no rolls come up as a 6 or exactly one roll comes up as a 6. This is\\
		$1 - \left(\binom{12}{0} \left(\frac{1}{6}\right)^0 \left(1-\frac{1}{6}\right)^{12} + \binom{12}{1} \left(\frac{1}{6}\right)^1 \left(1-\frac{1}{6}\right)^{11}\right)$.
		\item By linearity of expectation, $E[X] = E(I_1) + E(I_2) + \dots + E(I_{12})$.
		$=12 \cdot \frac{1}{6}$.


	\end{qparts}


\end{solution}

\subsection*{\probnum Fastest Draw in the Midwest [12 points]}

Suppose Grace has a standard deck of 52 cards. Grace expects she can draw all 52 cards in order (defined below) in 1300 draws. Yunsoo expects 1600 draws. Explain why Yunsoo is further away from the real expected value.

\textbf{Note:} The order of cards goes Ace, 2, 3, $\dots$, King and $\clubsuit, \diamondsuit, \heartsuit , \spadesuit$. If the next card in order is not drawn, then it is placed back into the deck at random. If the next card in order is drawn, then Grace sets it aside, removing it from the deck.

\textbf{Note:} The cards do not have to be selected consecutively. For example, \underline{A$\clubsuit$}, $3\diamondsuit,$ J$\spadesuit,$ \underline{2$\clubsuit$} is a valid start, and there would only be 50 cards left in the deck at this point.

\begin{solution}
	Let $X$ be the random variable for the number of draws it takes to draw the next card in order.\\
	Assume we start with the Ace of Clubs.\\
	$P(X = A\clubsuit) = \frac{1}{52}$\\
	$E[X] = \frac{1}{P(X)}=52$\\
	Similarly, since the right card won't get returned in the deck, it only takes 51 draws to get 2$\diamondsuit$.\\
	$P(X = 2\diamondsuit) = \frac{1}{51}$\\
	Thus, $E[X] = \frac{1}{P(X)}=52+51+50+$
	$\dots + 1$\\
	By the sum of an arithmetic series, $E[X] = \frac{52+1}{2} \cdot 52 = 1378$\\
	Therefore, the expected number of draws to draw the next card in order is 1378\\
	Grace is closer to the expected value than Yunsoo because 1300 is closer to 1378 than 1600 is to 1378.

\end{solution}
\end{document}
