\documentclass[12pt]{exam}

% essential packages
\usepackage{fullpage} % margin formatting
\usepackage{enumitem} % configure enumerate and itemize
\usepackage{amsmath, amsfonts, amssymb, mathtools} % math symbols
\usepackage{xcolor, colortbl} % colors, including in tables
\usepackage{makecell} % thicker \Xhline in table
\usepackage{graphicx} % images, resizing

% sometimes needed packages
\usepackage{hyperref} % hyperlinks
% \hypersetup{colorlinks=true, urlcolor=blue}
% \usepackage{logicproof} % natural deduction
\usepackage{tikz} % drawing graphs
\usetikzlibrary{positioning}
\usepackage{multicol}
% \usepackage{algpseudocode} % pseudocode

% paragraph formatting
\setlength{\parskip}{6pt}
\setlength{\parindent}{0cm}

% newline after Solution:
\renewcommand{\solutiontitle}{\noindent\textbf{Solution:}\par\noindent}

% less space before itemize/enumerate
\setlist{topsep=0pt}

% creates \filcl to grey out cells for groupwork grading
\newcommand{\filcl}{\cellcolor{gray!25}}

% creates \probnum to get the problem number
\newcounter{probnumcount}
\setcounter{probnumcount}{1}
\newcommand{\probnum}{\arabic{probnumcount}. \addtocounter{probnumcount}{1}}

% use roman numerals by default
\setlist[enumerate]{label={(\roman*)}}

% creates custom list environments for grading guidelines, question parts
\newlist{guidelines}{itemize}{1}
\setlist[guidelines]{label={}, left=0pt .. \parindent, nosep}
\newlist{gwguidelines}{enumerate}{1}
\setlist[gwguidelines]{label={(\roman*)}, nosep}
\newlist{qparts}{enumerate}{2}
\setlist[qparts]{label={(\alph*)}}
\newlist{qsubparts}{enumerate}{2}
\setlist[qsubparts]{label={(\roman*)}}
\newlist{stmts}{enumerate}{1}
\setlist[stmts]{label={(\roman*)}, nosep}
\newlist{pflist}{itemize}{4}
\setlist[pflist]{label={$\bullet$}, nosep}
\newlist{enumpflist}{enumerate}{4}
\setlist[enumpflist]{label={(\arabic*)}, nosep}

\printanswers

\newcommand{\prevhwnum}{9}
\newcommand{\hwnum}{10}

\begin{document}
\setcounter{probnumcount}{1}
\section*{Groupwork \hwnum{} Problems}

\subsection*{\probnum Circular Reasoning [15 points]}

Suppose we select $2n$ distinct points independently and uniformly at random on the border of a circle, and label them $p_1$ through $p_{2n}$ counter-clockwise (i.e. point $p_2$ is counter-clockwise from point $p_1$).

\begin{qparts}
    \item In the case where $n=2$, we have four distinct points on the circle. If we select two of these points uniformly at random and draw a line segment between them, then draw a line segment between the remaining two points, what is the probability that these line segments intersect?
    
    \textbf{Hint:} Consider the different cases corresponding to the point $p_1$ is paired with.
    
    \item Suppose we repeat the procedure in (a) where we select two points at random and draw a line segment between them. We'll call this line segment $\ell_1.$ We repeat this again with the $2n-2$ remaining points, creating a line segment $\ell_2$, etc., until we have drawn $n$ line segments: $\ell_1,\dots,\ell_n.$ After this procedure is completed, what is the expected number of intersections? Your answer should be in terms of $n.$
    
    \textbf{Hint:} Create an indicator random variable for each possible intersection and apply linearity of expectation.
    
    \textit{Note:} The number of intersections is the number of pairs $(\ell_i,\ell_j)$ of distinct line segments where $\ell_i$ and $\ell_j$ intersect.
\end{qparts}
\begin{solution}

\end{solution}

\subsection*{\probnum Open or Closed [20 points]}

\textit{Online Bayseian Inference} is a process where we repeatedly apply Bayes
rule to update our beliefs over time. Suppose we have a sensor that determines
whether a door is open or closed. 
If the door is open, the sensor reads it as open with probability 0.9. 
If the door is closed, the sensor reads it as closed with probability 0.7. Suppose the door starts in an unknown position, and has equal probability of being open or closed.

\begin{qparts}
    \item After one reading that the door is closed, what is the probability that the  door is actually closed?
    \item Before the second reading, we believe that the door is closed with the 
    probability found in part (a) (that is, we consider the probability that the door
    is closed to be the probability that we found the door is closed given our first reading). 
    Suppose we make another reading that the door  is closed. Now what is 
    the probability that the door is closed?
    \item On the third reading, the sensor reads that the door is open. What is 
    the probability that the door is actually closed, using the answer from part (b)
    as our initial probability for the door being closed?
\end{qparts}

\begin{solution}

\end{solution}


\end{document}
