\documentclass[12pt]{exam}

% essential packages
\usepackage{fullpage} % margin formatting
\usepackage{enumitem} % configure enumerate and itemize
\usepackage{amsmath, amsfonts, amssymb, mathtools} % math symbols
\usepackage{xcolor, colortbl} % colors, including in tables
\usepackage{makecell} % thicker \Xhline in table
\usepackage{graphicx} % images, resizing

% sometimes needed packages
\usepackage{hyperref} % hyperlinks
% \hypersetup{colorlinks=true, urlcolor=blue}
% \usepackage{logicproof} % natural deduction
\usepackage{tikz} % drawing graphs
\usetikzlibrary{positioning}
\usepackage{multicol}
\usepackage{algpseudocode} % pseudocode

% paragraph formatting
\setlength{\parskip}{6pt}
\setlength{\parindent}{0cm}

% newline after Solution:
\renewcommand{\solutiontitle}{\noindent\textbf{Solution:}\par\noindent}

% less space before itemize/enumerate
\setlist{topsep=0pt}

% creates \filcl to grey out cells for groupwork grading
\newcommand{\filcl}{\cellcolor{gray!25}}

% creates \probnum to get the problem number
\newcounter{probnumcount}
\setcounter{probnumcount}{1}
\newcommand{\probnum}{\arabic{probnumcount}. \addtocounter{probnumcount}{1}}

% use roman numerals by default
\setlist[enumerate]{label={(\roman*)}}

% creates custom list environments for grading guidelines, question parts
\newlist{guidelines}{itemize}{1}
\setlist[guidelines]{label={}, left=0pt .. \parindent, nosep}
\newlist{gwguidelines}{enumerate}{1}
\setlist[gwguidelines]{label={(\roman*)}, nosep}
\newlist{qparts}{enumerate}{2}
\setlist[qparts]{label={(\alph*)}}
\newlist{qsubparts}{enumerate}{2}
\setlist[qsubparts]{label={(\roman*)}}
\newlist{stmts}{enumerate}{1}
\setlist[stmts]{label={(\roman*)}, nosep}
\newlist{pflist}{itemize}{4}
\setlist[pflist]{label={$\bullet$}, nosep}
\newlist{enumpflist}{enumerate}{4}
\setlist[enumpflist]{label={(\arabic*)}, nosep}

\printanswers

\newcommand{\prevhwnum}{10}
\newcommand{\hwnum}{11}

\begin{document}
\section*{Individual Portion}

\subsection*{\probnum Big-$O$reo [15 points]}
Give the tightest big-O estimate for each of the following functions. Justify your answers.
\begin{qparts}
	\item $f(n) = (2^n + n^n)\cdot (n^3 + n \log n^n)$
	\item $g(n) = (n^n + n!)\cdot (n + 1)\hspace{0.1in} + \hspace{0.1in}(n^3 + 3^n)\cdot(\sqrt{n} + \log n)$
	\item $h(n) = (n^n + n^2)\cdot(n^n + n)\hspace{0.1in} + \hspace{0.1in}(\log 3 + n^n)\cdot(n^2 + n^n)$
\end{qparts}
\begin{solution}
	\begin{qparts}
		\item $n^n$ grows faster than $2^n$ and $n^3$ grows faster than $n\log n^n$. Thus, the tightest big-O estimate for $f(n)$ is $O(n^{n+3})$.
		\item $n^n$ grows faster than $n!$ and $3^n$ grows faster than $n^3$. Thus, the tightest big-O estimate for $g(n)$ is $O(n^{n+1})$.
		\item $n^n$ grows faster than $n^2$ and $n^n$ grows faster than $n$. Thus, the tightest big-O estimate for $h(n)$ is $O(n^{2n})$.
	\end{qparts}

\end{solution}

\subsection*{\probnum On the Run [20 points]}
Give the tightest big-O estimate for the number of operations (where an operation is arithmetic, a comparison, or an assignment) used in each of the following algorithms. \textbf{Explain your reasoning.}

\begin{qparts}

	\item \begin{algorithmic}
		\Function{doubleTrouble}{$a_1,\dots,a_N\in\mathbb{R}, j\in\mathbb{R}$}
		\State{$j \gets 1$}
		\For{$i\coloneqq 1\text{ to }N$}
		\If{$i = j$}
		\State{$j \gets 2j$}
		\EndIf
		\EndFor
		\State\Return{$j$}
		\EndFunction
	\end{algorithmic}

	\item \begin{algorithmic}
		\Function{sumSquares}{$N \in \mathbb{Z}^+$}
		\If{$N = 1$}
		\State\Return{$1$}
		\EndIf
		\State{$value \gets \textsc{sumSquares}(N - 1) + N^2$}
		\State\Return{$value$}
		\EndFunction
	\end{algorithmic}

	\item \begin{algorithmic}
		\Function{findLTMinProduct}{$a_1,\dots,a_N\in\mathbb{R}$}
		\State{$p\gets 203$}
		\For{$i\coloneqq 1\text{ to }N$}
		\For{$j\coloneqq 1\text{ to }N$}
		\If{$a_ia_j < p$}
		\State{$p\gets a_ia_j$}
		\EndIf
		\EndFor
		\EndFor
		\State{$numLTMinProduct\gets 0$}
		\For{$k\coloneqq 1\text{ to }N$}
		\If{$a_k < p$}
		\State{$numLTMinProduct \gets numLTMinProduct + 1$}
		\EndIf
		\EndFor
		\State\Return{$numLTMinProduct$}
		\EndFunction
	\end{algorithmic}

	\item \begin{algorithmic}
		\Function{subtractAndAdd}{$N\in\mathbb{Z}$}
		\While{$N > 0$}
		\If{$N$ is even}
		\State{$N \gets N - 3$}
		\EndIf
		\If{$N$ is odd}
		\State{$N \gets N + 1$}
		\EndIf
		\EndWhile
		\State\Return{$N$}
		\EndFunction
	\end{algorithmic}

	\item \begin{algorithmic}
		\Function{search}{$a_1,\dots,a_N \in \mathbb{R}, target \in \mathbb{R}$}
		\State{$left\gets 1$}
		\State{$right\gets N$}
		\While{$\text{True}$}
		\State{$mid\gets \lfloor \frac{left + right}{2} \rfloor$}
		\If{$a_{mid} = target$}
		\State\Return{$mid$}
		\EndIf
		\If{$right \leq left$}
		\State\Return{$-1$}
		\EndIf
		\If{$a_{mid} < target$}
		\State{$left\gets mid + 1$}
		\EndIf
		\If{$a_{mid} > target$}
		\State{$right\gets mid - 1$}
		\EndIf
		\EndWhile
		\EndFunction
	\end{algorithmic}
\end{qparts}

\begin{solution}
	\begin{qparts}
		\item O(DoubleTrouble) = O(N). This is because the loop is run N times, and the only operation inside the loop is an assignment.
		\item O(sumSquares) = O(N). This is because the function is called recursively N times, and the only operations inside the function are assignments and arithmetic operations.
		\item O(findLTMinProduct) = O($N^2$). This is because there are two nested loops that run N times each, and the only operations inside the loops are assignments and comparisons.
		\item O(subtractAndAdd) = O(N). This is because the while loop runs N times, and the only operations inside the loop are assignments and comparisons.
		\item O(search) = O(log N). This is because in this binary search the while loop runs log N times, and the only operations inside the loop are assignments and comparisons.
	\end{qparts}

\end{solution}


\subsection*{\probnum This one's bound to be fun! [18 points]}
You are given the following bounds on functions $f$ and $g$:
\begin{itemize}
	\item $f(x)$ is $O(203^xx^2)$ and $\Omega(3^x\log x)$
	\item $g(x)$ is $O(\frac{x!}{2^x})$ and $\Omega(4^x)$
\end{itemize}
Find the following, simplify your answer as much as possible.
\begin{qparts}
	\item Find the tightest big-O and big-$\Omega$ estimates that can be \textit{guaranteed} of $f(x)(g(x))^2$.
	\item Find the tightest big-O and big-$\Omega$ estimates that can be \textit{guaranteed} of $f(x)+g(x)$.
	\item Let $h(x)=f(x)-g(x)$. Prove or disprove that $h(x)$ is $\Omega(4^x)$.
\end{qparts}

\begin{solution}
	$\Omega$ is a lower bound, so we can use the lower bound of $f(x)$ and $g(x)$ to find the lower bound of $h_n(x)$.\\
	$O$ is an upper bound, so we can use the upper bound of $f(x)$ and $g(x)$ to find the upper bound of $h_n(x)$.
	\begin{qparts}
		\item Let $h_1(x)$ be $f(x)(g(x))^2$.\\
		Since $f(x)$ is $O(203^xx^2)$ and $g(x)$ is $O(\frac{x!}{2^x})$, we have that $h_1(x)$ is $O(203^xx^2\left(\frac{x!}{2^x}\right)^2)$.\\
		Since $f(x)$ is $\Omega(3^x\log x)$ and $g(x)$ is $\Omega(4^x)$, we have that $h_1(x)$ is $\Omega(3^x\log x\left(4^x\right)^2)$.\\
		% simplify
		Thus, the tightest big-O estimate for $h_1(x)$ is $O(203^xx^2\left(\frac{x!}{2^x}\right)^2)$ and the tightest big-$\Omega$ estimate for $h_1(x)$ is $\Omega(3^x\log x\left(4^x\right)^2)$.
		\item Let $h_2(x)$ be $f(x)+g(x)$.\\
		Since $f(x)$ is $O(203^xx^2)$ and $g(x)$ is $O(\frac{x!}{2^x})$, we have that $h_2(x)$ is $O(203^xx^2+\frac{x!}{2^x})$.\\
		Since $f(x)$ is $\Omega(3^x\log x)$ and $g(x)$ is $\Omega(4^x)$, we have that $h_2(x)$ is $\Omega(3^x\log x+4^x)$.\\
		Thus, the tightest big-O estimate for $h_2(x)$ is $O(203^xx^2)$ and the tightest big-$\Omega$ estimate for $h_2(x)$ is $\Omega(4^x)$.

		\item		Disproof by counterexample:\\
		\begin{tabular}{ll}
			Consider $f(x) = 4^x\log x + 2x$ and $g(x) = 4^x\log x$.                   \\
			Then $h(x) = f(x) - g(x) = 2x$.                                            \\
			Since $h(x) = 2x$ is $O(x)$, $h(x)$ is not guaranteed to be $\Omega(4^x)$. \\
			Therefore, $h(x)$ is not $\Omega(4^x)$.
		\end{tabular}
	\end{qparts}
\end{solution}

\pagebreak
\subsection*{\probnum Big Function Fun [16 points]}
Prove or disprove the following:
\begin{qparts}
	\item If $f(x)$ is $O(g(x))$ then $2^{f (x)}$ is $O(2^{g(x)}).$
	\item If $f(x)$ is $O(g(x))$ then $(f(x))^2$ is $O\big((g(x))^2\big).$
\end{qparts}
Note that in these proofs you do not need to use the definition of big-O, but can use the properties for combining mathematical functions covered in lecture.

\begin{solution}
	\begin{qparts}
		\item
		\begin{tabular}{ll}
			Disproof by counterexample:                                                             \\
			% consider fx = 10x, gx = x
			Consider $f(x) = 10x$ and $g(x) = x$.                                                   \\
			Then $f(x)$ is $O(g(x))$ since $10x$ is $O(x)$.                                         \\
			However, $2^{f(x)} = 2^{10x}$ is not $O(2^{g(x)}) = O(2^x)$.                            \\
			Since $2^{10x}$ grows exponentially faster than $2^x$, $2^{f(x)}$ is not $O(2^{g(x)})$. \\
			Therefore, the statement is false.
		\end{tabular}
		\item
		\begin{tabular}{ll}
			Proof by direct proof:                                                                    \\
			Assume $f(x)$ is $O(g(x))$.                                                               \\
			Then there exists a constant $c$ such that $f(x) \leq c \cdot g(x)$ for all $x \geq x_0$. \\
			Squaring both sides, we get $(f(x))^2 \leq c^2 \cdot (g(x))^2$ for all $x \geq x_0$.      \\
			Thus, $(f(x))^2$ is $O\big((g(x))^2\big)$.                                                \\
			Therefore, the statement is true.
		\end{tabular}
	\end{qparts}

\end{solution}


\subsection*{\probnum Roots and Shoots [16 points]}
Suppose $f$ satisfies $f(n) = 2f(\sqrt{n}) + \log_2 n$, whenever $n$ is a perfect square greater than 1, and additionally satisfies $f(2) = 1$.
\begin{qparts}
	\item Find $f(16)$.
	\item Find a big-O estimate for $g(m)$ where $g(m) = f(2^m).$

	\textbf{Hint:} Make the substitution $m = \log_2 n.$
	\item Find a big-O estimate for $f(n)$.
\end{qparts}

\begin{solution}
	\begin{qparts}
		\item We can divide the recurrence relation into smaller parts:
		\begin{align*}
			f(16) & = 2f(\sqrt{16}) + \log_2 16 \\
			      & = 2f(4) + 4                 \\
			      & = 2(2f(2) + 2) + 4          \\
			      & = 2(2(1) + 2) + 4           \\
			      & = 2(4) + 4                  \\
			      & = 8 + 4                     \\
			      & = 12
		\end{align*}
		Thus, $f(16) = 12$.
		\item Let $m = \log_2 n$. Then $n = 2^m$. We can rewrite the recurrence relation as:
		% only rewrite the recurrence relation once
		\begin{align*}
			f(2^m) & = 2f(\sqrt{2^m}) + \log_2 2^m \\
			       & = 2f(2^{m/2}) + m             \\
		\end{align*}
		We can see that $f(2^m) = 2f(2^{m/2}) + m$.
		To rewrite in terms of $g(m)$, we can substitute $g(m) = f(2^m)$:
		\begin{align*}
			g(m) & = 2g(m/2) + m
		\end{align*}
		By the Master Theorem with $a = 2$, $b = 2$, $d = 1$, and $f(n) = n$, $g(m) = O(m\log m)$ since $\frac{a}{b^d}=\frac{2}{2}=1$.

		\item Using the same substitution, since $m =  \log_2 n$ and $g(m) = O(m\log m)$, we have that $f(n) = O(\log_2 n \log \log_2 n)$.
	\end{qparts}
\end{solution}


\subsection*{\probnum GG Brown Laboratory [15 points]}
What is the tightest big-O bound on the runtime complexity of the following algorithm?
\begin{algorithmic}
	\Function{badsearch}{$n$}
	\If{$n\geq 1$}
	\State{$\textsc{badsearch}(\lfloor \frac n3\rfloor)$}
	\For{$i\coloneqq 1\text{ to }n$}
	\For{$j\coloneqq 1\text{ to }\lfloor \frac n2\rfloor$}
	\State{\textbf{print} ``Hello I am lost"}
	\EndFor
	\EndFor
	\State{$\textsc{badsearch}(\lfloor \frac n3\rfloor)$}
	\State{\textbf{print} ``Nevermind I got it"}
	\EndIf
	\EndFunction
\end{algorithmic}

\begin{solution}
	This is a recursive algorithm that calls itself twice with $\frac{n}{3}$ as the argument.
	The outer loop runs $n$ times and the inner loop runs $\frac{n}{2}$ times.
	The print statement runs $\frac{n^2}{2}$ times.
	Thus, we can write this as a recurrence relation:
	\[T(n) = 2T\left(\frac{n}{3}\right) + \frac{n^2}{2}\]
	Using the Master Theorem, we can see that $a = 2$, $b = 3$, $d=2$, and $f(n) = \frac{n^2}{2} = O(n^2)$.
	Since $\frac{a}{b^d} = \frac{2}{3^2} < 1$, we have that $T(n) = O(n^2)$.
\end{solution}
\end{document}
