\documentclass[12pt]{exam}

% essential packages
\usepackage{fullpage} % margin formatting
\usepackage{enumitem} % configure enumerate and itemize
\usepackage{amsmath, amsfonts, amssymb, mathtools} % math symbols
\usepackage{xcolor, colortbl} % colors, including in tables
\usepackage{makecell} % thicker \Xhline in table
\usepackage{graphicx} % images, resizing

% sometimes needed packages
\usepackage{hyperref} % hyperlinks
% \hypersetup{colorlinks=true, urlcolor=blue}
% \usepackage{logicproof} % natural deduction
% \usepackage{tikz} % drawing graphs
% \usetikzlibrary{positioning}
% \usepackage{multicol}
% \usepackage{algpseudocode} % pseudocode

% paragraph formatting
\setlength{\parskip}{6pt}
\setlength{\parindent}{0cm}

% newline after Solution:
\renewcommand{\solutiontitle}{\noindent\textbf{Solution:}\par\noindent}

% less space before itemize/enumerate
\setlist{topsep=0pt}

% creates \filcl to grey out cells for groupwork grading
\newcommand{\filcl}{\cellcolor{gray!25}}

% creates \probnum to get the problem number
\newcounter{probnumcount}
\setcounter{probnumcount}{1}
\newcommand{\probnum}{\arabic{probnumcount}. \addtocounter{probnumcount}{1}}

% use roman numerals by default
\setlist[enumerate]{label={(\roman*)}}

% creates custom list environments for grading guidelines, question parts
\newlist{guidelines}{itemize}{1}
\setlist[guidelines]{label={}, left=0pt .. \parindent, nosep}
\newlist{gwguidelines}{enumerate}{1}
\setlist[gwguidelines]{label={(\roman*)}, nosep}
\newlist{qparts}{enumerate}{2}
\setlist[qparts]{label={(\alph*)}}
\newlist{qsubparts}{enumerate}{2}
\setlist[qsubparts]{label={(\roman*)}}
\newlist{stmts}{enumerate}{1}
\setlist[stmts]{label={(\roman*)}, nosep}
\newlist{pflist}{itemize}{4}
\setlist[pflist]{label={$\bullet$}, nosep}
\newlist{enumpflist}{enumerate}{4}
\setlist[enumpflist]{label={(\arabic*)}, nosep}

\printanswers

\newcommand{\prevhwnum}{5}
\newcommand{\hwnum}{6}

\begin{document}
\pagebreak
\setcounter{probnumcount}{1}
\section*{Groupwork \hwnum{} Problems}


\subsection*{\probnum Multiple Multiples [12 points]}
Let $a,b\in \mathbb{Z}$. Show that $7a - 8b$ is a multiple of 5 if and only if $19a - 21b$ is a multiple of 5.

\begin{solution}
	Let $a,b\in \mathbb{Z}$. We want to show that $7a - 8b$ is a multiple of 5 if and only if $19a - 21b$ is a multiple of 5.\\
	In other words, we want to show that $7a - 8b \equiv 0 \pmod{5}$ if and only if $19a - 21b \equiv 0 \pmod{5}$.\\
	i.e. in logic notation, we want to show that $(7a - 8b \equiv 0 \pmod{5}) \iff (19a - 21b \equiv 0 \pmod{5})$.\\
	We will show this by proving both directions of the biconditional.\\
	\begin{tabular}{ll}
		LHS:                                                                           \\
		Assume $7a - 8b \equiv 0 \pmod{5}$                         &                   \\
		$7a - 8b\pmod{5} \equiv 2a-3b\pmod{5} \equiv 0\pmod{5}$    & definition of mod \\
		$2a-3b=5k$ for some $k\in\mathbb{Z}$                       & definition of mod \\
		$2a=5k+3b$                                                 & algebra           \\
		$4a=10k+6b$                                                & algebra           \\
		$19a-21b\pmod{5} \equiv 4a-1b\pmod{5}$                     & definition of mod \\
		$10k+6b-1b\pmod{5} \equiv 10k+5b\pmod{5} \equiv 0\pmod{5}$ & algebra           \\
	\end{tabular}\\
	Therefore, $19a - 21b \equiv 0 \pmod{5}$.\\
	\begin{tabular}{ll}
		RHS:                                                                           \\
		Similarly, assume $19a - 21b \equiv 0 \pmod{5}$            &                   \\
		$19a - 21b\pmod{5} \equiv 4a-1b\pmod{5} \equiv 0\pmod{5}$  & definition of mod \\
		$4a-1b=5k$ for some $k\in\mathbb{Z}$                       & definition of mod \\
		$b=4a-5k$                                                  & algebra           \\
		$7a - 8b\pmod{5} \equiv 7a-8b\pmod{5}\equiv 2a-3b\pmod{5}$ & definition of mod \\
		$2a-3b\pmod{5} \equiv 2a-3(4a-5k)\pmod{5}$                 & substitution      \\
		$2a-3(4a-5k)\pmod{5} \equiv 2a-12a+15k\pmod{5}$            & algebra           \\
		$2a-12a+15k\pmod{5} \equiv -10a+15k\pmod{5}$               & algebra           \\
		$-10a+15k\pmod{5} \equiv 0\pmod{5}$                        & algebra           \\
	\end{tabular}\\
	Therefore, $7a - 8b \equiv 0 \pmod{5}$.\\
	Since we have shown both directions of the biconditional, we have shown that $7a - 8b$ is a multiple of 5 if and only if $19a - 21b$ is a multiple of 5.
\end{solution}

\subsection*{\probnum Rapidly Rising [18 points]}
For this problem, we will say a function $f\colon \mathbb{Z}^+ \to \mathbb{Z}^+$ is ``rapidly rising'' if:
$$ \forall x_1, x_2 \in \mathbb{Z}^+ \ [x_1 < x_2 \to 2f(x_1) < f(x_2)] $$

\begin{qparts}
	\item Prove that $f(x) = 3^x$ is rapidly rising.

	\textbf{Hint:} It may be easier to show $f(x_2) > 2f(x_1)$ than the other way around.

	\item Is a rapidly rising function always one-to-one? Is a one-to-one function from $\mathbb{Z}^+\to\mathbb{Z}^+$ always rapidly rising? Is a one-to-one function (again from $\mathbb{Z}^+\to\mathbb{Z}^+$) always strictly increasing? Briefly explain your answer; a formal proof is not necessary but is encouraged.

	\textbf{Note:} $f\colon\mathbb{N}\to\mathbb{N}$ is strictly increasing if $f(x_1)<f(x_2)$ whenever $x_1<x_2.$
	\item Prove that, for any rapidly rising function $f$, it must \textbf{not} be onto.
\end{qparts}


\begin{solution}
  a) Let $f(x) = 3^x$. We want to show that $f(x)$ is rapidly rising.\\
  Let $x_1, x_2 \in \mathbb{Z}^+$ such that $x_1 < x_2$.\\
  We want to show that $2f(x_1) < f(x_2)$.\\
  We have $f(x_1) = 3^{x_1}$ and $f(x_2) = 3^{x_2}$.\\
  We want to show that $2\cdot 3^{x_1} < 3^{x_2}$.\\
  Since for $f(x)$, the base is 3, thus for every increase in $x$, the value of $f(x)$ is multiplied by 3.\\
  Thus, no matter what the value of $x_1$ is, $3^{x_1}$ will always be less than $3^{x_2}$ for any $x_2 > x_1$.\\
  Therefore, $2\cdot 3^{x_1} < 3^{x_2}$.\\
  Since we have shown that $2f(x_1) < f(x_2)$, we have shown that $f(x) = 3^x$ is rapidly rising.\\
  b) A rapidly rising function is not always one-to-one.\\
  A one-to-one function from $\mathbb{Z}^+\to\mathbb{Z}^+$ is not always rapidly rising.\\
  A one-to-one function from $\mathbb{Z}^+\to\mathbb{Z}^+$ is not always strictly increasing.\\
  A rapidly rising function is not always one-to-one because a rapidly rising function only guarantees that $2f(x_1) < f(x_2)$ for $x_1 < x_2$. It does not guarantee that $f(x_1) \neq f(x_2)$ for $x_1 \neq x_2$.\\
  A one-to-one function from $\mathbb{Z}^+\to\mathbb{Z}^+$ is not always rapidly rising because a one-to-one function only guarantees that $f(x_1) \neq f(x_2)$ for $x_1 \neq x_2$. It does not guarantee that $2f(x_1) < f(x_2)$ for $x_1 < x_2$.\\
  A one-to-one function from $\mathbb{Z}^+\to\mathbb{Z}^+$ is not always strictly increasing because a one-to-one function only guarantees that $f(x_1) \neq f(x_2)$ for $x_1 \neq x_2$. It does not guarantee that $f(x_1) < f(x_2)$ for $x_1 < x_2$.\\
  c) Let $f$ be a rapidly rising function. We want to show that $f$ must not be onto.\\
  A function $f$ is onto if for every $y\in\mathbb{Z}^+$, there exists an $x\in\mathbb{Z}^+$ such that $f(x) = y$.\\
  We will show that $f$ must not be onto by contradiction.\\
  Assume for the sake of contradiction that $f$ is onto.\\
  Since $f$ is onto, for every $y\in\mathbb{Z}^+$, there exists an $x\in\mathbb{Z}^+$ such that $f(x) = y$.\\
  Let $x_1, x_2 \in \mathbb{Z}^+$ such that $x_1 < x_2$.\\
  We have $f(x_1) < f(x_2)$ because $f$ is rapidly rising.\\
  Since $f$ is onto, for every $y\in\mathbb{Z}^+$, there exists an $x\in\mathbb{Z}^+$ such that $f(x) = y$.\\
  Let $y = f(x_2)$.\\
  Since $f$ is onto, there exists an $x\in\mathbb{Z}^+$ such that $f(x) = f(x_2)$.\\
  Since $f$ is one-to-one, $x = x_2$.\\
  Since $x_1 < x_2$, $f(x_1) < f(x_2)$.\\
  This is a contradiction because $f(x_1) < f(x_2)$ but $f(x_1) = f(x_2)$.\\
  Therefore, $f$ must not be onto.\\
  Since we have shown that $f$ must not be onto, we have shown that for any rapidly rising function $f$, it must not be onto.
\end{solution}


\end{document}
