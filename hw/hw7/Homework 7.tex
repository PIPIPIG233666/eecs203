\documentclass[12pt]{exam}

\usepackage{tikz} % Used for Groupwork 7 PHP

% essential packages
\usepackage{fullpage} % margin formatting
\usepackage{enumitem} % configure enumerate and itemize
\usepackage{amsmath, amsfonts, amssymb, mathtools} % math symbols
\usepackage{xcolor, colortbl} % colors, including in tables
\usepackage{makecell} % thicker \Xhline in table
\usepackage{graphicx} % images, resizing

% sometimes needed packages
\usepackage{hyperref} % hyperlinks
% \hypersetup{colorlinks=true, urlcolor=blue}
% \usepackage{logicproof} % natural deduction
% \usepackage{tikz} % drawing graphs
% \usetikzlibrary{positioning}
% \usepackage{multicol}
% \usepackage{algpseudocode} % pseudocode

% paragraph formatting
\setlength{\parskip}{6pt}
\setlength{\parindent}{0cm}

% newline after Solution:
\renewcommand{\solutiontitle}{\noindent\textbf{Solution:}\par\noindent}

% less space before itemize/enumerate
\setlist{topsep=0pt}

% creates \filcl to grey out cells for groupwork grading
\newcommand{\filcl}{\cellcolor{gray!25}}

% creates \probnum to get the problem number
\newcounter{probnumcount}
\setcounter{probnumcount}{1}
\newcommand{\probnum}{\arabic{probnumcount}. \addtocounter{probnumcount}{1}}

% use roman numerals by default
\setlist[enumerate]{label={(\roman*)}}

% creates custom list environments for grading guidelines, question parts
\newlist{guidelines}{itemize}{1}
\setlist[guidelines]{label={}, left=0pt .. \parindent, nosep}
\newlist{gwguidelines}{enumerate}{1}
\setlist[gwguidelines]{label={(\roman*)}, nosep}
\newlist{qparts}{enumerate}{2}
\setlist[qparts]{label={(\alph*)}}
\newlist{qsubparts}{enumerate}{2}
\setlist[qsubparts]{label={(\roman*)}}
\newlist{stmts}{enumerate}{1}
\setlist[stmts]{label={(\roman*)}, nosep}
\newlist{pflist}{itemize}{4}
\setlist[pflist]{label={$\bullet$}, nosep}
\newlist{enumpflist}{enumerate}{4}
\setlist[enumpflist]{label={(\arabic*)}, nosep}

\printanswers

\newcommand{\prevhwnum}{6}
\newcommand{\hwnum}{7}

\begin{document}
%%%%%%%%%%%%%%% TITLE PAGE %%%%%%%%%%%%%%%
\title{EECS 203: Discrete Mathematics\\
	Winter 2024\\
	Homework \hwnum{}}
\date{}
\author{}
\maketitle
\vspace{-50pt}
\begin{center}
	\huge Due \textbf{Thursday, Mar. 21}, 10:00 pm\\
	\Large No late homework accepted past midnight.\\
	\vspace{10pt}
	\large Number of Problems: $8+2$
	\hspace{3cm}
	Total Points: $100+18$
\end{center}
\vspace{25pt}
\begin{itemize}
	\item \textbf{Match your pages!} Your submission time is when you upload the file, so the time you take to match pages doesn't count against you.
	\item Submit this assignment (and any regrade requests later) on Gradescope.
	\item Justify your answers and show your work (unless a question says otherwise).
	\item By submitting this homework, you agree that you are in compliance with the Engineering Honor Code and the Course Policies for 203, and that you are submitting your own work.
	\item Check the syllabus for full details.
\end{itemize}
\newpage
%%%%%%%%%%%%%%% TITLE PAGE %%%%%%%%%%%%%%% 

\section*{Individual Portion}

\subsection*{\probnum Growing your Growth Mindset [5 points]}

\begin{qparts}
	\item Watch the linked video about developing a growth mindset. This is a different video than the one you saw in lecture.
	\item Rewrite the last two fixed mindset statements as growth mindset statements.
	\item Write down one of your recurring fixed mindset thoughts, then write a thought you can replace it with that reflects a growth mindset.
\end{qparts}

\textbf{Video:} \href{https://tinyurl.com/eecs203growthMindset}{Developing a Growth Mindset (tinyurl.com/eecs203growthMindset)}

\textbf{What to submit:} Your three pairs of fixed and growth mindset statements
(the two from the table, and one that you came up with on your own).

\begin{center}
	\begin{tabular}{ |p{18em}|p{18em}| }
		\hline
		\textbf{Fixed Mindset Statement} & \textbf{Growth Mindset Statement}                                                                                                                             \\
		\hline

		When I have to ask for help or get called on in lecture, I get anxious and feel like people will think I’m not smart.

		                                 & The question I have is likely the same question someone else in lecture may have. It’s important for me to ask so I can better understand what I am learning.
		\\ \hline

		I’m jealous of other people’s success.

		                                 & I am inspired and encouraged by other people’s success. They show me what is possible.

		\\ \hline

		I didn’t score as high on the exam as I expected. I’m not going to do well in this class and should drop it.

		                                 & I learned from my mistakes on exam 1, and exam 2 will be a new opportunity for me to practice what I’ve learned.

		\\ \hline

		This class is hard for me, so I am not fit for this major.
		                                 & [FILL IN YOUR OWN]
		\\ \hline

		Either I'm good at Discrete Math, or I'm not.
		                                 & [FILL IN YOUR OWN]
		\\ \hline

		[FILL IN YOUR OWN]               & [FILL IN YOUR OWN]                                                                                                                                            \\
		\hline
	\end{tabular}
\end{center}

\begin{solution}
	I am learning to be better at Discrete Math, and I will get better with practice.\\
	I can be good at Discrete Math if I put in the effort to learn and understand the material.\\
	I am not smart enough to understand this material.\\
	The content is hard but it is possible for me to understand it with practice and effort as well as learning from other people.
\end{solution}


\subsection*{\probnum Sketchy Compositions [15 points]}
Consider $f \colon X \to Y $ and $g \colon Y \to X$. \textbf{Prove or disprove} each of the following statements.

\begin{qparts}
	\item If  $ f \circ g $ is one-to-one, then $ g $ must be one-to-one.
	\item If $ g \circ f $ is one-to-one, then $ g $ must be one-to-one.
\end{qparts}

\begin{solution}
	a) Proof by contradiction:\\
	\begin{tabular}{ll}
		Seeking contradiction, assume the negation of the statement:                                        \\
		$f \circ g$ is one-to-one $\land g$ is $\lnot$ one-to-one.                                          \\
		$\exists x_1, x_2 \in X$ such that $x_1 \neq x_2$ and $g(x_1) = g(x_2)$ & Definition of one-to-one  \\
		Since $x_1 \neq x_2$ $\land$ $f(g(x_1)) = f(g(x_2))$                    & Definition of $f \circ g$ \\
		$f \circ g$ is not one-to-one                                           & Definition of one-to-one  \\
		This contradicts the assumption that $f \circ g$ is one-to-one.                                     \\
		Therefore, if $f \circ g$ is one-to-one, then $g$ must be one-to-one.                               \\
		Thus, the original statement holds by proof by contradiction.                                       \\
	\end{tabular}
	b) Proof by contrapositive:\\
	\begin{tabular}{ll}
		The contropostive is $g\lnot$ one-to-one $\rightarrow g \circ f\lnot$ one-to-one.                           \\
		Asseme $g\lnot $ one-to-one:                                                                                \\
		$\exists y_1, y_2 \in Y [y_1 \neq y_2\land g(y_1) = g(y_2)]$                    & Definition of one-to-one  \\
		Thus, $\exists f(y_1), f(y_2)$$[g(f(y_1)) = g(f(y_2)) \land f(y_1)\neq f(y_2)]$ & Definition of $g \circ f$ \\
		Which means $g \circ f\lnot $ one-to-one.                                       & Definition of one-to-one  \\
		Thus the original statement holds by contrapositive.                                                        \\
	\end{tabular}

\end{solution}

\subsection*{\probnum Flippy Function Fun! [15 points]}

A function $f\colon A\to A$ is said to be \textit{flippy} if for all $a\in A,$ $f(f(a))=a.$ \textbf{Prove or disprove} each of the following statements

\begin{qparts}
	\item If $f\colon A\to A$ is flippy, then $f$ is bijective. (Either prove $f$ is both onto and one-to-one using their respective definitions, or provide a counterexample.)
	\item If $f\colon A\to A$ and $g\colon A\to A$ are flippy, then $f\circ g$ must be flippy.
\end{qparts}

\begin{solution}
	a) Proof by direct proof:\\
	\begin{tabular}{ll}
		Assume arbitary $f\colon A\to A$ is flippy.           \\
		We need to show that $f$ is both onto and one-to-one. \\
		Proof one-to-one:                                     \\
		Let $a,b$ be arbitary elements $\in A$                \\
		Assume $f(a) = f(b)$                                  \\
		Then $f(f(a)) = f(f(b))$                              \\
		Since $f$ is flippy, $f(f(a)) = a$ and $f(f(b)) = b$  \\
		Thus $a = b$                                          \\
		Therefore, $f$ is one-to-one.                         \\
		Proof onto:                                           \\
		Let $b$ be an arbitary element $\in A$                \\
		Let $a = f(b)$                                        \\
		Then $f(a) = f(f(b))$                                 \\
		Since $f$ is flippy, $f(f(b)) = b$                    \\
		Thus $f(a) = b$                                       \\
		Therefore, $f$ is onto.                               \\
		Thus, $f$ is bijective.                               \\
	\end{tabular}
	\\b) Disproof by counterexample:\\
	\begin{tabular}{ll}
		Will show $f: A\to A \land g: A\to A$ are flippy $\rightarrow f\circ g$ is not flippy for set $A = \{1,2,3,4\}$ \\
		Consider flippy function $f: A\to A$                                                                            \\
		Let $f(1) = 1, f(2) = 4, f(3) = 3, f(4) = 2$                                                                    \\
		Consider flippy function $g: A\to A$                                                                            \\
		Let $g(1) = 4, g(2) = 3, g(3) = 2, g(4) = 1$                                                                    \\
		Consider $x=1$                                                                                                  \\
		$f(g(f(g(1)))) = f(g(f(4))) = f(g(2)) = f(3) = 3\neq 1$                                                         \\
		Thus $f\circ g$ is not flippy.                                                                                  \\
		Therefore, the original statement is disproof by counterexample.                                                \\
	\end{tabular}

\end{solution}


\subsection*{\probnum A Hairy Situation [12 points]}
Assume that nobody on Earth has more than 1,000,000 hairs on their head. Assume that the population of New York City in 2024 is 8,468,000 people. As of 2024, what is the maximum number of people in New York City that we can guarantee all have the same number of hairs on their heads?

Your explanation should use the Pigeonhole Principle. Make sure to state what the pigeons and holes are, as well as how many of each you have.

\begin{solution}
	By the Pigeonhole Principle, we can guarantee that at least 9 people in New York City have the same number of hairs on their heads.\\
	Pigeons: People in New York City, 8,468,000\\
	Holes: Number of hairs on a person's head, could be 0 to 1,000,000, thus 1,000,001\\
	$\lceil \frac{8,468,000}{1,000,001} \rceil = 9$\\
\end{solution}


\subsection*{\probnum A Pairy Situation [14 points]}
Suppose that $52$ integers are chosen among the set of natural numbers less than $100$. In other words, suppose that $52$ integers are chosen from $\{0 , 1, 2, 3, \dots, 99\}$. \textbf{Prove or disprove} that there must exist at least one pair of integers among those chosen whose difference is equal to 7.

Your proof or disproof should use the Pigeonhole Principle. Make sure to state what the pigeons and holes are, as well as how many of each you have.

\begin{solution}
	Proof:\\
	Pigeons: 52 integers chosen from $\{0 , 1, 2, 3, \dots, 99\}$\\
	Holes: \\
	We have the following $7\cdot 7+2=51$ grid of numbers, where each row represents a set of numbers that differ by $7$: \\
	% no duplication
	$\{0,7\},\{1,8\},\{2,9\},\{3,10\},\{4,11\},\{5,12\},\{6,13\}$\\
	$\{14,21\},\{15,22\},\{16,23\},\{17,24\},\{18,25\},\{19,26\},\{20,27\}$\\
	$\{28,35\},\{29,36\},\{30,37\},\{31,38\},\{32,39\},\{33,40\},\{34,41\}$\\
	$\{42,49\},\{43,50\},\{44,51\},\{45,52\},\{46,53\},\{47,54\},\{48,55\}$\\
	$\{56,63\},\{57,64\},\{58,65\},\{59,66\},\{60,67\},\{61,68\},\{62,69\}$\\
	$\{70,77\},\{71,78\},\{72,79\},\{73,80\},\{74,81\},\{75,82\},\{76,83\}$\\
	$\{84,91\},\{85,92\},\{86,93\},\{87,94\},\{88,95\},\{89,96\},\{90,97\}$\\
	$\{98\},\{99\}$\\
	Since we have 52 pigeons and 51 holes, by the Pigeonhole Principle, at least one hole must contain at least two pigeons.\\
	Thus, there must exist at least one pair of integers among those chosen whose difference is equal to 7.\\
\end{solution}


\subsection*{\probnum Super Sets [15 points]}
Let $A$ be the set of prime numbers less than $203$. The universe of discourse is $\mathbb{R}$. State whether each of the following sets are empty, finite but nonempty, countably infinite, or uncountable. Briefly justify your answers.

\begin{qparts}
	\item $\mathbb{Z} \times \mathbb{Z}$
	\item $(\mathbb{Z} \times \mathbb{Z}) - (\mathbb{Q} \times \mathbb{Q})$
	\item $\mathbb{R} - \mathbb{Q}$
	\item $\mathbb{Q} - \mathbb{R}$
	\item $A \cap \mathbb{Q}$
	\item $\overline{A} \cap \overline{\mathbb{Q}}$
\end{qparts}

\begin{solution}
	a) Countably infinite\\  Each element in $\mathbb{Z} \times \mathbb{Z}$ can be mapped to a unique element in $\mathbb{N}$, thus $\mathbb{Z} \times \mathbb{Z}$ is countably infinite.\\
	b) Empty\\ $\mathbb{Z} \subseteq \mathbb{Q}$, thus $\mathbb{Z} \times \mathbb{Z} \nsubseteq \mathbb{Q} \times \mathbb{Q}$, thus $(\mathbb{Z} \times \mathbb{Z}) - (\mathbb{Q} \times \mathbb{Q})$ is empty.\\
	c) Uncountable infinite\\ $\mathbb{R} - \mathbb{Q} =$Irrational numbers, which are uncountable infinite.\\
	, thus $\mathbb{R} - \mathbb{Q}$ is uncountable infinite.\\
	d) Empty\\ $\mathbb{Q} \subseteq \mathbb{R}$, thus $\mathbb{Q} - \mathbb{R}$ is empty.\\
	e) Finite but nonempty\\ Since $A \subseteq \mathbb{Q}$, $A \cap \mathbb{Q}$ is finite but nonempty because $A$ is all prime numbers less than 203, which is finite but nonempty.\\
	finite but nonempty.\\
	f) Uncountable infinite\\ By compliment, $\overline{A} \cap \overline{\mathbb{Q}} = \mathbb{R} - A \cap \mathbb{R} - \mathbb{Q}$, which is the set of union of real numbers that are not prime numbers less than 203 and irrational numbers, which is uncountable infinite.\\
\end{solution}


\subsection*{\probnum Cardinal Construction [12 points]}
For each part, give \textit{uncountable} sets $A$ and $B$ such that $A - B$ is
\begin{qparts}
	\item uncountable.
	\item countably infinite.
	\item finite but nonempty.
	\item empty.
\end{qparts}\

\begin{solution}
	a) $A=\mathbb{R}_{\ge 0}, B= [0,1]$\\ $A-B=\mathbb{R}_{>1}$\\
	b) $A=\mathbb{R}\cup \mathbb{Z}, B=\mathbb{R}$\\ $A-B=\mathbb{Z}$\\
	c) $A=\mathbb{R}_{\ge 0}, B=\mathbb{R}_{\le 0}$\\ $A-B=0$\\
	d) $A=\mathbb{R}, B=\mathbb{R}$\\ $A-B=\emptyset$

\end{solution}


\subsection*{\probnum Interesting Intervals [12 points]}
Prove that $|[0,3]| = |(2, 5)\cup (6,7)|$. If you construct functions in your solution with certain properties, you may assert that they have those properties without proof.

\begin{solution}
	Let $A = [0,3]$ and $B = (2, 5)\cup (6,7)$.\\
	Let $f(x)=\frac{x}{5}+2$ for $x\in A$ be one-to-one and $g(x)=\frac{x}{1000}-2.0001$ for $x\in B$ be one-to-one\\
	Thus, $f:A\to B$ and $g:B\to A$ are both one-to-one.\\
	Since $f$ is one-to-one, $|A|\leq |B|$.\\
	Since $g$ is one-to-one, $|B|\leq |A|$.\\
	Thus, $|A|=|B|$ by Schröder-Bernstein Theorem.\\
	Therefore $|[0,3]| = |(2, 5)\cup (6,7)|$.
\end{solution}

\end{document}
