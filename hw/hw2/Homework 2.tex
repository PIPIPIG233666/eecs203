\documentclass[12pt]{exam}

% essential packages
\usepackage{fullpage} % margin formatting
\usepackage{enumitem} % configure enumerate and itemize
\usepackage{amsmath, amsfonts, amssymb, mathtools} % math symbols
\usepackage{xcolor, colortbl} % colors, including in tables
\usepackage{makecell} % thicker \Xhline in table
\usepackage{graphicx} % images, resizing

% sometimes needed packages
\usepackage{hyperref} % hyperlinks
% \hypersetup{colorlinks=true, urlcolor=blue}
\usepackage{logicproof} % natural deduction
% \usepackage{tikz} % drawing graphs
% \usetikzlibrary{positioning}
% \usepackage{multicol}
% \usepackage{algpseudocode} % pseudocode

% paragraph formatting
\setlength{\parskip}{6pt}
\setlength{\parindent}{0cm}

% newline after Solution:
\renewcommand{\solutiontitle}{\noindent\textbf{Solution:}\par\noindent}

% less space before itemize/enumerate
\setlist{topsep=0pt}

% creates \filcl to grey out cells for groupwork grading
\newcommand{\filcl}{\cellcolor{gray!25}}
\newcommand{\divides}{\,|\,}

% creates \probnum to get the problem number
\newcounter{probnumcount}
\setcounter{probnumcount}{1}
\newcommand{\probnum}{\arabic{probnumcount}. \addtocounter{probnumcount}{1}}

% use roman numerals by default
\setlist[enumerate]{label={(\roman*)}}

% creates custom list environments for grading guidelines, question parts
\newlist{guidelines}{itemize}{1}
\setlist[guidelines]{label={}, left=0pt .. \parindent, nosep}
\newlist{gwguidelines}{enumerate}{1}
\setlist[gwguidelines]{label={(\roman*)}, nosep}
\newlist{qparts}{enumerate}{2}
\setlist[qparts]{label={(\alph*)}}
\newlist{qsubparts}{enumerate}{2}
\setlist[qsubparts]{label={(\roman*)}}
\newlist{stmts}{enumerate}{1}
\setlist[stmts]{label={(\roman*)}, nosep}
\newlist{pflist}{itemize}{4}
\setlist[pflist]{label={$\bullet$}, nosep}
\newlist{enumpflist}{enumerate}{4}
\setlist[enumpflist]{label={(\arabic*)}, nosep}

\printanswers

\newcommand{\prevhwnum}{1}
\newcommand{\hwnum}{2}

\begin{document}
%%%%%%%%%%%%%%% TITLE PAGE %%%%%%%%%%%%%%%
\title{EECS 203: Discrete Mathematics\\
	Winter 2024\\
	Homework \hwnum{}}
\date{}
\author{}
\maketitle
\vspace{-50pt}
\begin{center}
	\huge Due \textbf{Thursday, Feb. 1st}, 10:00 pm\\
	\Large No late homework accepted past midnight.\\
	\vspace{10pt}
	\large Number of Problems: $8+2$
	\hspace{3cm}
	Total Points: $100+40$
\end{center}
\vspace{25pt}
\begin{itemize}
	\item \textbf{Match your pages!} Your submission time is when you upload the file, so the time you take to match pages doesn't count against you.
	\item Submit this assignment (and any regrade requests later) on Gradescope.
	\item Justify your answers and show your work (unless a question says otherwise).
	\item By submitting this homework, you agree that you are in compliance with the Engineering Honor Code and the Course Policies for 203, and that you are submitting your own work.
	\item Check the syllabus for full details.
\end{itemize}
\newpage
%%%%%%%%%%%%%%% TITLE PAGE %%%%%%%%%%%%%%% 

\section*{Individual Portion}

\subsection*{\probnum Negation Transformation [12 points]}
Find the negation of each statement below. If your thought process involves intermediate steps, show them. If not, simply writing the negation is sufficient.

Your answer should not contain the original proposition. That is, you shouldn't negate it as ``It is not the case that ...'' or something similar.
\begin{qparts}
	\item Every student in this course is enrolled in exactly one Discussion section.
	\item There is a student in this class who is not on Gradescope or not on Piazza.
	\item For all integers $a$ and $b$, if $a+b > 0$, then $a-b < 0$.
	\item For every irrational number $x$, there is a rational number $y$ such that $x^y$ is rational.
\end{qparts}

\begin{solution}
	a) Some student in this course
	is not enrolled in exactly one Discussion section.\\
	b) Every student in this class is on Gradescope and on Piazza.\\
	c) There exists an integer $a$ and $b$, if $a - b > 0$, then $a + b < 0$.\\
	d) This propsition translates to $\forall x \exists y ({x^y}\in \mathbb{Q})$, thus, to negate this, we flip everything. $\exists x \forall y ({x^y}\in \mathbb{I}).$ Therefore, there exists irrational number x, for all y, such that ${x^y}$ is irrational.
\end{solution}


\subsection*{\probnum Not It [12 points]}
Negate the following statements so that all negation symbols immediately precede predicates. Make sure to show all intermediate steps.

\textbf{Note:} $\lnot (P(x)\vee Q(x))$ would not be considered fully simplified since the negation ($\lnot$) does not immediately come before $P(x)$ or $Q(x).$ However, $\lnot P(x) \vee \lnot Q(x)$ is fully simplified, for example.

\begin{qparts}
	\item $\forall y [ \exists x P(x,y) \vee \forall x Q(x, y) ]$
	\item $\exists x \forall y [ R(x,y) \rightarrow R(y, x) ]$
\end{qparts}

\begin{solution}
	\begin{tabular}{ll}
		a)
		$\neg [\forall y [ \exists x P(x,y) \vee \forall x Q(x, y) ]]$                          \\
		$\equiv \exists y[\forall x \lnot P(x,y)\land \exists x\lnot Q(x,y)]$ & Demorgan's Laws \\
	\end{tabular}

	\begin{tabular}{ll}
		b)
		$\neg [\exists x \forall y [ R(x,y) \rightarrow R(y, x) ]]$                                    \\
		$\equiv \forall x \exists y  [\lnot[R(x,y) \rightarrow R(y, x) ]]$ & Demorgan's Laws           \\
		$\equiv \forall x \exists y[R(x,y)\land \lnot R(y,x)]$             & Implication Breakout Rule \\
	\end{tabular}
\end{solution}


\subsection*{\probnum Order's Up! [12 points]}
Let $P(x,y)$ be the statement ``customer $x$ has ordered dish $y$," where the domain for $x$ consists of all customers and for $y$ consists of all dishes at a restaurant.  Express each of these propositions in logic.
\begin{qparts}
	\item Some customer has ordered some dish at this restaurant.
	\item Some customer has ordered all of the dishes at this restaurant.
	\item Each customer has ordered at least one dish at this restaurant.
	\item Some dish at this restaurant has been ordered by all customers.
	\item Each dish at this restaurant has been ordered by at least one customer.
	\item All customers have ordered every dish at this restaurant.
	\item Some dish at this restaurant has been ordered by a customer.
	\item Every dish at this restaurant has been ordered by every customer.
\end{qparts}

\begin{solution}
	a) $\exists x \exists y P(x,y)$
	\\b) $\exists x \forall y P(x,y)$
	\\c) $\forall x \exists y P(x,y)$
	\\d) $\exists y \forall x P(x,y)$
	\\e) $\forall y \exists x P(x,y)$
	\\f) $\forall x \forall y P(x,y)$
	\\g) $\exists y \exists x P(x,y)$
	\\h) $\forall y \forall x P(x,y)$

\end{solution}


\subsection*{\probnum Sports Statements  [12 points]}
Let $I(x)$ be the statement ``$x$ has a favorite sport" and $C(x,y)$ be the statement ``$x$ and $y$ have the same favorite sport," where the domain for the variables $x$ and $y$ consists of all students in your class. Use quantifiers and the logical connectives you learned in lecture to express each of the statements below.

\textbf{Hint:} You can use an $=$ sign to compare people.
\begin{qparts}
	\item Someone in your class does not have a favorite sport.
	\item No one in the class has the same favorite sport as Chloe.
	\item Everyone except one student in your class has a favorite sport.
\end{qparts}


\begin{solution}
 a) $\exists x \lnot I(x)$
 \\ b) $\forall x \lnot [x\neq Chloe \rightarrow \lnot C(Chloe,x)]$
 \\ c) $\exists x \forall y [x\neq y \iff I(y)]$
\end{solution}


\subsection*{\probnum Quantifier Quandary [12 points]}
For each of the propositions below, write the negation, and determine whether the original proposition is true or if its negation is true. Your negation cannot contain the logical ``not" symbol ($\lnot$), but you may use the not-equals sign ($\ne$). The domain of discourse is all real numbers. \textbf{Briefly justify your answers.}
\begin{qparts}
	\item $\exists x (x^3 = -1)$
	\item $\forall x (2x > x)$
	\item $\exists x \forall y (x+y = 0)$
	\item $\forall x \exists y (x+y = 0)$
\end{qparts}
\begin{solution}
 a) negation: $\forall x(x^3 \neq -1)$ consider $x =-1$, the negation is false, therefore the original proposition is true.
 \\ b) negation: $\exists x(2x < x)$ consider $x$ to be any negative real numbers, the original proposition is false, therefore the negation is true.
 \\ c) negation: $\forall x \exists y (x+y \neq 0)$ the original proposition means that there is only one $x$ for all $y$ that will make $x+y=0$ true. Consider $x=0, y=1,2,100$, the original proposition is false, and the original proposition is true.
 \\ d) negation: $\exists x \forall y (x+y \neq 0)$ same idea as c), however, consider $x=0,y=0$, the negation is false, therefore the original is true.
\end{solution}

\subsection*{\probnum Even Stevens [8 points]}

Prove that if $n$ is an even integer, then $\frac{n^2}{2}$ is also an even integer.

\begin{solution}
  \begin{tabular}{ll}
  Proof: Assume $n$ is an even integer &premise\\
  $n = 2k$, $k$ is an integer &definition of even\\
  $\frac{n^2}{2}$ \\ $\equiv \frac{{2k}^2}{2}$ \\$\equiv \frac {4k^2}{2}$ \\ $\equiv 2(k^2)$ \\
  Since $k$ is integer, $k^2$ is also integer, $2(k^2)$ is 2 times an integer\\
  Thus, $n^2/2$ is also an even integer\\
\end{tabular}
\end{solution}

\subsection*{\probnum To Prove or Not To Prove [16 points]}

\textbf{Prove or disprove} each of the following statements where the domain of discourse is all real numbers.
\begin{qparts}
	\item For all $x,$ $x^2>0.$
	\item There exists $x$ such that $x\le 0$ and $2x > x.$
	\item There exists $x$ such that for all $y,$ $x^2+y^2>203.$
	\item There exists $x$ such that for all $y,$ $(x+y)^2>203.$
\end{qparts}

\begin{solution}
 \begin{tabular}{ll}
   a) Disproof: consider $x=0$, $x^2=0$, therefore, the proposition is false\\
   b) Proof: $\lnot [\exists x (x \leq 0 \land 2x > x)]\equiv \forall x (x > 0 \lor 2x \leq x)$\\
   Case 1: if $x > 0$ is true, the negation is true. \\
   Case 2: if $x \leq 0$ is true\\
   $2x \leq x$ is true, the negation is true. Thus, the original proposition is false.\\
   c) Proof: Let $x=1000$, y be an arbitary real number.\\
   $1000^2 + y^2 > 203$ is true. Therefore, $\exists x \forall y (x^2+y^2>203)$ is true.\\
   d) Proof: Let $x=1000$, y be an arbitary real number.\\
   $(1000+y)^2 >203$ is true. Therefore, $\exists x \forall y (x+y)^2>203$ is true.
 
 \end{tabular}

\end{solution}


\subsection*{\probnum Mixed Quantifiers Proof [16 points]}

For this problem, let the domain of discourse be positive integers.
\begin{qparts}
	\item Consider the following predicate:
	$$ P(x, z) := (z > x) \land (x \,|\, z) \land (4 \nmid z) $$
	Let $x = 10$. Find the three smallest values of $z$ which satisfy $P(10, z).$

	\item Now prove the following proposition:
	$$ \forall x [ 4 \nmid x \to \exists z\, P(x, z) ]$$
\end{qparts}

\textbf{Note:} The statement $a\,|\, b$ means ``$a$ divides $b$," i.e. there exists some integer $q$ such that $b=aq.$ Similarly, $a\nmid b$ means ``$a$ does not divide $b.$"

\begin{solution}
a) To find $z$ such that $P(10,z)$ is true, $(z > 10) \land (10 \mid z) \land (4 \nmid z )$ is true, $z$ has to be greater than 10, divide 10 but not 4. The three smallest values of $z$ can be $30, 50, 70$.
\\ b) Proof: Let $x$ be an arbitary positive integer, if $x$ does not divide 4, let $z=3x$, as $z$ is a multiple of $x$ that does not form 4 with factors of x, the proposition is true.
\end{solution}

\pagebreak
\section*{Grading of Groupwork \prevhwnum{}}
Using the solutions and Grading Guidelines, grade your Groupwork \prevhwnum{} Problems:
\begin{itemize}
	\item Use the table below to grade your past groupwork submission and calculate scores.
	\item While grading, mark up your past submission. Include this with the table when you submit your grading.
	\item Write whether your submission achieved each rubric item. If it didn't achieve one, say why not.
	\item For extra credit, write positive comment(s) about your work.
	\item You don't have to redo problems correctly, but it is recommended!
	\item See ``All About Groupwork" on Canvas for more detailed guidance, and what to do if you change groups.
\end{itemize}

\begin{center}
	\resizebox{\textwidth}{!}{\begin{tabular}{| c | c | c | c | c | c | c | c | c | c | c | c | c |}
			\hline
			          & (i)    & (ii)   & (iii)  & (iv)   & (v)    & (vi)   & (vii)  & (viii) & (ix)   & (x)    & (xi)   & Total:          \\
			\hline
			Problem 1 &   +2     & +4       & +0       &+2        &+4        &+4        &+2        & \filcl & \filcl & \filcl & \filcl & 18/20 \\
			\Xhline{1.25pt}
			Total:    & \filcl & \filcl & \filcl & \filcl & \filcl & \filcl & \filcl & \filcl & \filcl & \filcl & \filcl & 18/20 \\
			\hline
		\end{tabular}}
\end{center}
\begin{solution}
  I stated the premise wrong when I said exactly one of the four is lying instead when I meant telling the truth. I also did not state the different cases because I learnt about this easier solution of finding the two that contradict each other first.
\end{solution}

\pagebreak
\setcounter{probnumcount}{1}
\section*{Groupwork \hwnum{} Problems}

\subsection*{\probnum Bézout's Identity [20 points]}

In number theory, there's a simple yet powerful theorem called Bézout's identity, which states that for any two integers $a$ and $b$ (with $a$ and $b$ not both zero) there exist two integers $r$ and $s$ such that $ar+bs=\gcd(a,b).$ Use Bézout's identity to prove the following statements (you may assume all variables are integers):

\begin{qparts}
	\item If $d\divides a$ and $d\divides b,$ then $d\divides \gcd(a,b).$
	\item If $a\divides bc$ and $\gcd(a,b)=1,$ then $a\divides c.$
\end{qparts}

\noindent
\textbf{Note:} $\gcd$ is short for ``greatest common divisor," so the value of $\gcd(a,b)$ is the largest integer that evenly divides $a$ and $b.$ You won't need to apply this definition, just know that $\gcd(a,b)$ is an integer.


\begin{solution}
\begin{tabular}{ll}
  a) Assume $d \mid a \land d \mid b$ & premise\\
  $a=kd \land b=jd$, k and j are arbitary integers &definition of divide\\
  $kdr+jds=gcd(a,b)$ &Bezout's Identity\\
  $gcd(a,b)=d(kr+js)$ &factoring \\
  Since $kr+js$ is also an integer, thus $d \mid gcd(a,b)$ &Bezout's Identity\\
  Thus, $d \mid gcd(a,b)$.
\end{tabular}
\end{solution}


\subsection*{\probnum Proposition Michigan [20 points]}
Translate each of the following English statements into logic. You may define predicates as necessary.

\textbf{Note:} Your predicates should not trivialize the problem.

\begin{qparts}
	\item Each pair of students at UMich has at least two mutual friends at UMich. The domain of discourse is all students at UMich.

	\item Nobody knows everyone's Wolverine Access password except the Wolverine Access administrators, who know all passwords. The domain of discourse is all people who have a Wolverine Access account (the administrators have Wolverine Access accounts).
\end{qparts}

\begin{solution}

\end{solution}

\end{document}
