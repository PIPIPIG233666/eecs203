\documentclass[12pt]{exam}

% essential packages
\usepackage{fullpage} % margin formatting
\usepackage{enumitem} % configure enumerate and itemize
\usepackage{amsmath, amsfonts, amssymb, mathtools} % math symbols
\usepackage{xcolor, colortbl} % colors, including in tables
\usepackage{makecell} % thicker \Xhline in table
\usepackage{graphicx} % images, resizing

% sometimes needed packages
\usepackage{hyperref} % hyperlinks
% \hypersetup{colorlinks=true, urlcolor=blue}
% \usepackage{logicproof} % natural deduction
% \usepackage{tikz} % drawing graphs
% \usetikzlibrary{positioning}
% \usepackage{multicol}
% \usepackage{algpseudocode} % pseudocode

% paragraph formatting
\setlength{\parskip}{6pt}
\setlength{\parindent}{0cm}

% newline after Solution:
\renewcommand{\solutiontitle}{\noindent\textbf{Solution:}\par\noindent}

% less space before itemize/enumerate
\setlist{topsep=0pt}

% creates \filcl to grey out cells for groupwork grading
\newcommand{\filcl}{\cellcolor{gray!25}}

% creates \probnum to get the problem number
\newcounter{probnumcount}
\setcounter{probnumcount}{1}
\newcommand{\probnum}{\arabic{probnumcount}. \addtocounter{probnumcount}{1}}

% use roman numerals by default
\setlist[enumerate]{label={(\roman*)}}

% creates custom list environments for grading guidelines, question parts
\newlist{guidelines}{itemize}{1}
\setlist[guidelines]{label={}, left=0pt .. \parindent, nosep}
\newlist{gwguidelines}{enumerate}{1}
\setlist[gwguidelines]{label={(\roman*)}, nosep}
\newlist{qparts}{enumerate}{2}
\setlist[qparts]{label={(\alph*)}}
\newlist{qsubparts}{enumerate}{2}
\setlist[qsubparts]{label={(\roman*)}}
\newlist{stmts}{enumerate}{1}
\setlist[stmts]{label={(\roman*)}, nosep}
\newlist{pflist}{itemize}{4}
\setlist[pflist]{label={$\bullet$}, nosep}
\newlist{enumpflist}{enumerate}{4}
\setlist[enumpflist]{label={(\arabic*)}, nosep}

\printanswers

\newcommand{\prevhwnum}{5}
\newcommand{\hwnum}{6}

\begin{document}
%%%%%%%%%%%%%%% TITLE PAGE %%%%%%%%%%%%%%%
\title{EECS 203: Discrete Mathematics\\
  Winter 2024\\
  Homework \hwnum{}}
\date{}
\author{}
\maketitle
\vspace{-50pt}
\begin{center}
  \huge Due \textbf{Thursday, Mar. 14th}, 10:00 pm\\
\Large No late homework accepted past midnight.\\
\vspace{10pt}
\large Number of Problems: $7+2$
\hspace{3cm}
Total Points: $100+30$
\end{center}
\vspace{25pt}
\begin{itemize}
    \item \textbf{Match your pages!} Your submission time is when you upload the file, so the time you take to match pages doesn't count against you.
    \item Submit this assignment (and any regrade requests later) on Gradescope. 
    \item Justify your answers and show your work (unless a question says otherwise).
    \item By submitting this homework, you agree that you are in compliance with the Engineering Honor Code and the Course Policies for 203, and that you are submitting your own work.
    \item Check the syllabus for full details.
\end{itemize}
\newpage
%%%%%%%%%%%%%%% TITLE PAGE %%%%%%%%%%%%%%% 

\section*{Individual Portion}

\subsection*{\probnum Mod Warm-up [12 points]}

Find the integer $a$ such that
\begin{qparts}
\item $a\equiv 58 \pmod{18}$ and $0 \leq a\leq 17$;
\item $a \equiv -142 \pmod{7}$ and $0 \leq a \leq 6$;
\item $a \equiv 17 \pmod{29}$ and $-14 \leq a \leq 14$;
\item $a \equiv -11 \pmod{21}$ and $110 \leq a \leq 130$.
\end{qparts}
Show your intermediate steps or briefly explain your process to justify your work.

\begin{solution}

\end{solution}

\subsection*{\probnum Multiple Modular Madness [14 points]}
For each of the questions below, answer ``always," ``sometimes," or ``never," then explain your answer. Your explanation should justify why you chose the answer you did, but does not have to be a rigorous proof.

\textbf{Hint:} Recall that if $a\equiv b\pmod{m}$ then there exists an integer $k$ such that $a=b+mk.$
\begin{qparts}
    \item Suppose $a \equiv 2 \pmod{21}$. When is $a \equiv 2 \pmod{7}$?
    \item Suppose $b \equiv 2 \pmod{7}$. When is $b \equiv 2 \pmod{21}$?
    \item Suppose $c \equiv 5 \pmod{8}$. When is $c \equiv 4 \pmod{16}$?
    \item Suppose $d \equiv 3 \pmod{21}$. When is $d \equiv 0 \pmod{6}$?
\end{qparts}

\begin{solution}

\end{solution}

\subsection*{\probnum How Low Can You Go? [12 points]}
Suppose $a\equiv 3\pmod{10}$ and $b\equiv 8\pmod{10}$. In each part, find $c$ such that $0\leq c\leq 9$ and
\begin{qparts}
    \item $c \equiv 14a^2 - b^3 \pmod{10}$
    \item $c \equiv b^{15} - 99 \pmod{10}$
    \item $c\equiv a^{97} \pmod{10}$
\end{qparts}
Show your work! You should be doing the arithmetic/making substitutions \textbf{without using a calculator}. Your work must \textbf{not} include numbers above 100.

\begin{solution}

\end{solution}

\subsection*{\probnum Be There or Be Square [16 points]}

Prove that if $n$ is an odd integer, then $n^2 \equiv 1 \pmod 8.$

\textbf{Note:} You \textbf{cannot} use the fact that all integers are equivalent to one of 0-7 (mod 8) without proof.

\begin{solution}

\end{solution}


\subsection*{\probnum Functions and Fakers [16 points]}
Determine if each of the examples below are functions or not.
\begin{itemize}
    \item If it is not a function, explain why not.
    \item If it is a function, state whether or not it is bijective, and briefly justify your answer.
\end{itemize}
All domains and codomains are given as intended.

\begin{qparts}
    \item $f \colon \mathbb{R}^{\times} \to \mathbb{R}^{\times}$ such that $f(x) = x^{-1}.$

    \textbf{Note:} The set $\mathbb{R}^{\times}$ is the set $\mathbb{R}-\{0\}.$ Additionally, recall that $x^{-1}=\frac 1x.$
    
    \item $g \colon \mathbb{R} \to \mathbb{R}$ such that $g(x) = y$ iff $y \leq x.$
    \item $h \colon \textbf{U-M Courses} \to \{ \text{EECS}, \text{MATH} \}$ which maps each class to its department.
    \item $k \colon \textbf{U-M Courses} \to \mathbb{N}$ which maps each class to its course number
\end{qparts}
For example, $h(\text{EECS 203}) = \text{EECS}$ and $k(\text{EECS 203}) = 203.$

\textbf{Note:} For the purpose of parts (c) and (d), two courses are considered ``equal" if and only if they have the same department and course number. In particular, cross-listed courses are treated as distinct elements of $\textbf{U-M Courses}.$

\begin{solution}

\end{solution}


\subsection*{\probnum Fantastic Functions [18 points]}
For each of the functions below, determine whether it is (i) one-to-one, (ii) onto. Prove your answers.
\begin{qparts}
    \item $f \colon \mathbb{R} \to \mathbb R - \mathbb R^-,\ f(x) = e^{2x + 1}.$

    \item $g \colon \mathbb{R} - \left\{-\frac{2}{5}\right\} \to \mathbb{R} - \left\{\frac{3}{5}\right\},\ g(x) = \frac{3x - 1}{5x + 2}.$

    \item$h \colon \mathbb{Z} \times \mathbb {Z} \to \mathbb{Z},\ h(m, n) = |m|-|n|$
\end{qparts}


\begin{solution}

\end{solution}


\subsection*{\probnum Comp$\circ$sition$($Functions$)$ [12 points]}
For each of the following pairs of functions $f$ and $g$, find $f \circ g$ and $g \circ f$. Make sure to include the domain and codomain of each composed function you give. If either can't be computed, explain why.
\begin{qparts}
    \item $f\colon \mathbb{N} \to \mathbb{Z}^{+},\; f(x) = x + 1$
    
    $g\colon\mathbb{Z}^{+}\to\mathbb{N},\; g(x) = x^2 - 1$
    
    \item $f\colon\mathbb{Z}\to\mathbb{R},\; f(x) = \left(\frac{3}{2} x + 3\right)^{3}$
    
    $g\colon\mathbb{R} \rightarrow \mathbb{R}_{\geq 0},\; g(x) = |x|$

    \textbf{Note:} $\mathbb{R}_{\geq 0}$ is the set of real numbers greater than or equal to 0.
\end{qparts}


\begin{solution}

\end{solution}


\pagebreak
\section*{Grading of Groupwork \prevhwnum{}}
Using the solutions and Grading Guidelines, grade your Groupwork \prevhwnum{} Problems:
\begin{itemize}
    \item Use the table below to grade your past groupwork submission and calculate scores.
    \item While grading, mark up your past submission. Include this with the table when you submit your grading.
    \item Write whether your submission achieved each rubric item. If it didn't achieve one, say why not.
    \item For extra credit, write positive comment(s) about your work.
    \item You don't have to redo problems correctly, but it is recommended!
    \item See ``All About Groupwork" on Canvas for more detailed guidance, and what to do if you change groups.
\end{itemize}

\begin{center}
\resizebox{\textwidth}{!}{\begin{tabular}{| c | c | c | c | c | c | c | c | c | c | c | c | c |}
\hline
 & (i) & (ii) & (iii) & (iv) & (v) & (vi) & (vii) & (viii) & (ix) & (x) & (xi) & Total:\\
\hline
Problem 1 & & & & & & &\filcl &\filcl &\filcl & \filcl& \filcl& \hspace{1cm}/12\\
\hline 
Problem 2 & & & & & & &\filcl &\filcl &\filcl & \filcl& \filcl& \hspace{1cm}/18\\
\Xhline{1.25pt}
Total: &\filcl &\filcl &\filcl &\filcl &\filcl &\filcl &\filcl &\filcl & \filcl& \filcl& \filcl&\hspace{1cm}/30\\
\hline
\end{tabular}}
\end{center}

\pagebreak
\setcounter{probnumcount}{1}
\section*{Groupwork \hwnum{} Problems}


\subsection*{\probnum Multiple Multiples [12 points]}
Let $a,b\in \mathbb{Z}$. Show that $7a - 8b$ is a multiple of 5 if and only if $19a - 21b$ is a multiple of 5.

\begin{solution}

\end{solution}

\subsection*{\probnum Rapidly Rising [18 points]}
For this problem, we will say a function $f\colon \mathbb{Z}^+ \to \mathbb{Z}^+$ is ``rapidly rising'' if:
$$ \forall x_1, x_2 \in \mathbb{Z}^+ \ [x_1 < x_2 \to 2f(x_1) < f(x_2)] $$

\begin{qparts}
    \item Prove that $f(x) = 3^x$ is rapidly rising.

    \textbf{Hint:} It may be easier to show $f(x_2) > 2f(x_1)$ than the other way around.
    
    \item Is a rapidly rising function always one-to-one? Is a one-to-one function from $\mathbb{Z}^+\to\mathbb{Z}^+$ always rapidly rising? Is a one-to-one function (again from $\mathbb{Z}^+\to\mathbb{Z}^+$) always strictly increasing? Briefly explain your answer; a formal proof is not necessary but is encouraged.

    \textbf{Note:} $f\colon\mathbb{N}\to\mathbb{N}$ is strictly increasing if $f(x_1)<f(x_2)$ whenever $x_1<x_2.$
    \item Prove that, for any rapidly rising function $f$, it must \textbf{not} be onto.
\end{qparts}


\begin{solution}

\end{solution}


\end{document}