\documentclass[12pt]{exam}

% essential packages
\usepackage{fullpage} % margin formatting
\usepackage{enumitem} % configure enumerate and itemize
\usepackage{amsmath, amsfonts, amssymb, mathtools} % math symbols
\usepackage{xcolor, colortbl} % colors, including in tables
\usepackage{makecell} % thicker \Xhline in table
\usepackage{graphicx} % images, resizing

% sometimes needed packages
\usepackage{hyperref} % hyperlinks
% \hypersetup{colorlinks=true, urlcolor=blue}
% \usepackage{logicproof} % natural deduction
% \usepackage{tikz} % drawing graphs
% \usetikzlibrary{positioning}
% \usepackage{multicol}
% \usepackage{algpseudocode} % pseudocode

% paragraph formatting
\setlength{\parskip}{6pt}
\setlength{\parindent}{0cm}

% newline after Solution:
\renewcommand{\solutiontitle}{\noindent\textbf{Solution:}\par\noindent}

% less space before itemize/enumerate
\setlist{topsep=0pt}

% creates \filcl to grey out cells for groupwork grading
\newcommand{\filcl}{\cellcolor{gray!25}}

% creates \probnum to get the problem number
\newcounter{probnumcount}
\setcounter{probnumcount}{1}
\newcommand{\probnum}{\arabic{probnumcount}. \addtocounter{probnumcount}{1}}

% use roman numerals by default
\setlist[enumerate]{label={(\roman*)}}

% creates custom list environments for grading guidelines, question parts
\newlist{guidelines}{itemize}{1}
\setlist[guidelines]{label={}, left=0pt .. \parindent, nosep}
\newlist{gwguidelines}{enumerate}{1}
\setlist[gwguidelines]{label={(\roman*)}, nosep}
\newlist{qparts}{enumerate}{2}
\setlist[qparts]{label={(\alph*)}}
\newlist{qsubparts}{enumerate}{2}
\setlist[qsubparts]{label={(\roman*)}}
\newlist{stmts}{enumerate}{1}
\setlist[stmts]{label={(\roman*)}, nosep}
\newlist{pflist}{itemize}{4}
\setlist[pflist]{label={$\bullet$}, nosep}
\newlist{enumpflist}{enumerate}{4}
\setlist[enumpflist]{label={(\arabic*)}, nosep}

\printanswers

\newcommand{\prevhwnum}{5}
\newcommand{\hwnum}{6}

\begin{document}
%%%%%%%%%%%%%%% TITLE PAGE %%%%%%%%%%%%%%%
\title{EECS 203: Discrete Mathematics\\
	Winter 2024\\
	Homework \hwnum{}}
\date{}
\author{}
\maketitle
\vspace{-50pt}
\begin{center}
	\huge Due \textbf{Thursday, Mar. 14th}, 10:00 pm\\
	\Large No late homework accepted past midnight.\\
	\vspace{10pt}
	\large Number of Problems: $7+2$
	\hspace{3cm}
	Total Points: $100+30$
\end{center}
\vspace{25pt}
\begin{itemize}
	\item \textbf{Match your pages!} Your submission time is when you upload the file, so the time you take to match pages doesn't count against you.
	\item Submit this assignment (and any regrade requests later) on Gradescope.
	\item Justify your answers and show your work (unless a question says otherwise).
	\item By submitting this homework, you agree that you are in compliance with the Engineering Honor Code and the Course Policies for 203, and that you are submitting your own work.
	\item Check the syllabus for full details.
\end{itemize}
\newpage
%%%%%%%%%%%%%%% TITLE PAGE %%%%%%%%%%%%%%% 

\section*{Individual Portion}

\subsection*{\probnum Mod Warm-up [12 points]}

Find the integer $a$ such that
\begin{qparts}
	\item $a\equiv 58 \pmod{18}$ and $0 \leq a\leq 17$;
	\item $a \equiv -142 \pmod{7}$ and $0 \leq a \leq 6$;
	\item $a \equiv 17 \pmod{29}$ and $-14 \leq a \leq 14$;
	\item $a \equiv -11 \pmod{21}$ and $110 \leq a \leq 130$.
\end{qparts}
Show your intermediate steps or briefly explain your process to justify your work.

\begin{solution}
	a) $58\pmod{18} = 4$, so $a=4$.\\
	b) $-142\pmod{7} = -2$, so $a=-2+7=5$.\\
	c) $17\pmod{29} = 17$, so $a=17-29=-12$.\\
	d) $-11\pmod{21} = -11$, so $a=-11+21\cdot 6=115$.
\end{solution}

\subsection*{\probnum Multiple Modular Madness [14 points]}
For each of the questions below, answer ``always," ``sometimes," or ``never," then explain your answer. Your explanation should justify why you chose the answer you did, but does not have to be a rigorous proof.

\textbf{Hint:} Recall that if $a\equiv b\pmod{m}$ then there exists an integer $k$ such that $a=b+mk.$
\begin{qparts}
	\item Suppose $a \equiv 2 \pmod{21}$. When is $a \equiv 2 \pmod{7}$?
	\item Suppose $b \equiv 2 \pmod{7}$. When is $b \equiv 2 \pmod{21}$?
	\item Suppose $c \equiv 5 \pmod{8}$. When is $c \equiv 4 \pmod{16}$?
	\item Suppose $d \equiv 3 \pmod{21}$. When is $d \equiv 0 \pmod{6}$?
\end{qparts}

\begin{solution}
	a) Always. If $a\equiv 2\pmod{21}$, then $a=2+21k$ for some arbitary integer $k$. Then $a=2+7(3k)$, so $a\equiv 2\pmod{7}$.\\
	b) Sometimes. If $b\equiv 2\pmod{7}$, then $b=2+7k$ for some arbitary integer $k$. Then $b=2+21(\frac{1}{3}k)$, so $b$ is not necessarily congruent to 2 modulo 21. \\
	c) Never. If $c\equiv 5\pmod{8}$, then $c=5+8k$ for some arbitary integer $k$. Suppose $c\equiv 4\pmod{16}$, then $c=4+16j$ for some other arbitary integer $j$.\\
	Then $5+8k=4+16j$, so $8k+1=16j$. This implies that $16k$ is odd, which is a contradiction.\\
	Therefore, $c$ is never congruent to 4 modulo 16.\\
	d) Sometimes. If $d\equiv 3\pmod{21}$, then $d=3+21k$ for some arbitary integer $k$.\\
	Suppose $d\equiv 0\pmod{6}$, then $d=6j$ for some arbitary integer $j$.\\
	Then $3+21k=6j$, so $7k=2j-1$. This holds as long as $2j-1$ is a multiple of 7, and would be false when $j=k=0$.

\end{solution}

\subsection*{\probnum How Low Can You Go? [12 points]}
Suppose $a\equiv 3\pmod{10}$ and $b\equiv 8\pmod{10}$. In each part, find $c$ such that $0\leq c\leq 9$ and
\begin{qparts}
	\item $c \equiv 14a^2 - b^3 \pmod{10}$
	\item $c \equiv b^{15} - 99 \pmod{10}$
	\item $c\equiv a^{97} \pmod{10}$
\end{qparts}
Show your work! You should be doing the arithmetic/making substitutions \textbf{without using a calculator}. Your work must \textbf{not} include numbers above 100.

\begin{solution}
	Since $a\equiv 3\pmod{10}$, $a=3+10k$ for some integer $k$. Similarly, $b=8+10l$ for some integer $l$.\\
	a) $c\equiv 14(3+10k)^2 - (8+10l)^3\pmod{10}$.\\
	$\equiv 14(3)^2 - 8^3\pmod{10}$.\\
	$\equiv 14(9) - 8^2\cdot 8\pmod{10}$.\\
	$\equiv 4\cdot 9 - 4\cdot 8\pmod{10}$.\\
	$\equiv 36 - 32\pmod{10}$.\\
	$\equiv 4\pmod{10}$.\\
	So $c=4$.\\
	b) $c\equiv (8+10l)^{15} - 99\pmod{10}$.\\
	$\equiv 8^{15} - 99\pmod{10}$.\\
	$\equiv (8^3)^5 - 99\pmod{10}$.\\
	$\equiv (8^2\cdot 8)^5 - 99\pmod{10}$.\\
	$\equiv (4\cdot 8)^5 - 99\pmod{10}$.\\
	$\equiv 32^5 - 99\pmod{10}$.\\
	$\equiv 2^5 - 99\pmod{10}$.\\
	$\equiv 32 - 99\pmod{10}$.\\
	$\equiv 2 - 9\pmod{10}$.\\
	$\equiv -7\pmod{10}$.\\
	So $c=-7+10=3$.\\
	c) $c\equiv (3+10k)^{97}\pmod{10}$.\\
	$\equiv 3^{100}\cdot 3^{-3}\pmod{10}$.\\
	$\equiv (3^5)^{20}\cdot 3^{-3}\pmod{10}$.\\
	$\equiv (3^3\cdot 3^2)^{20}\cdot 3^{-3}\pmod{10}$.\\
	$\equiv (27\cdot 9)^{20}\cdot 3^{-3}\pmod{10}$.\\
	$\equiv (7\cdot 9)^{20}\cdot 3^{-3}\pmod{10}$.\\
	$\equiv 3^{20}\cdot 3^{-3}\pmod{10}$.\\
	$\equiv (3^5)^4\cdot 3^{-3}\pmod{10}$.\\
	$\equiv 3^4\cdot 3^{-3}\pmod{10}$.\\
	$\equiv 3$.\\
	So $c=3$.
\end{solution}

\subsection*{\probnum Be There or Be Square [16 points]}

Prove that if $n$ is an odd integer, then $n^2 \equiv 1 \pmod 8.$

\textbf{Note:} You \textbf{cannot} use the fact that all integers are equivalent to one of 0-7 (mod 8) without proof.

\begin{solution}
	Proof: \\
	Proof by cases.\\
	\begin{tabular}{l l}
		Assume $n$ is an arbitrary odd integer.           & assumption               \\
		$n=2k+1$ for some integer $k$, $k\in \mathbb{Z}$  & Definition of odd        \\
		$n^2=(2k+1)^2=4k^2+4k+1=4(k^2+k)+1$.              & substitution             \\
		Case 1: $k$ is even.                              &                          \\
		$k=2m$ for some integer $m$, $m\in \mathbb{Z}$.   & Definition of even       \\
		$n^2=4(4m^2+2m)+1=16m^2+8m+1$.                    & substitution             \\
		$n^2\equiv 1\pmod 8$.                             & Definition of congruence \\
		Case 2: $k$ is odd.                               &                          \\
		$k=2m+1$ for some integer $m$, $m\in \mathbb{Z}$. & Definition of odd        \\
		$n^2=4(4m^2+6m+2)+1=16m^2+24m+9$.                 & substitution             \\
		$n^2\equiv 9\pmod 8\equiv 1\pmod 8$.              & Definition of congruence \\
		WLOG, all cases have been exhausted.              &                          \\
		Therefore, $n^2\equiv 1\pmod 8$.                  &                          \\
	\end{tabular}

\end{solution}


\subsection*{\probnum Functions and Fakers [16 points]}
Determine if each of the examples below are functions or not.
\begin{itemize}
	\item If it is not a function, explain why not.
	\item If it is a function, state whether or not it is bijective, and briefly justify your answer.
\end{itemize}
All domains and codomains are given as intended.

\begin{qparts}
	\item $f \colon \mathbb{R}^{\times} \to \mathbb{R}^{\times}$ such that $f(x) = x^{-1}.$

	\textbf{Note:} The set $\mathbb{R}^{\times}$ is the set $\mathbb{R}-\{0\}.$ Additionally, recall that $x^{-1}=\frac 1x.$

	\item $g \colon \mathbb{R} \to \mathbb{R}$ such that $g(x) = y$ iff $y \leq x.$
	\item $h \colon \textbf{U-M Courses} \to \{ \text{EECS}, \text{MATH} \}$ which maps each class to its department.
	\item $k \colon \textbf{U-M Courses} \to \mathbb{N}$ which maps each class to its course number
\end{qparts}
For example, $h(\text{EECS 203}) = \text{EECS}$ and $k(\text{EECS 203}) = 203.$

\textbf{Note:} For the purpose of parts (c) and (d), two courses are considered ``equal" if and only if they have the same department and course number. In particular, cross-listed courses are treated as distinct elements of $\textbf{U-M Courses}.$

\begin{solution}
	a) $f$ is a bijective function. It is a function because for every $x\in\mathbb{R}^{\times}$, there is a unique $x^{-1}\in\mathbb{R}^{\times}$. It is bijective because it is both 1-1 and onto. This is because $\forall x\in\mathbb{R}^{\times}\exists y\in\mathbb{R}^{\times}$ such that $f(x)=f(y)\rightarrow x=y$ and $\forall y\in\mathbb{R}^{\times}\exists x\in\mathbb{R}^{\times}$ such that $f(x)=y$.\\
	b) $g$ is not a function. It is not a function because for every $x\in\mathbb{R}$, there is not a unique $y\in\mathbb{R}$ such that $g(x)=y$. Consider $x=0$, then $g(0)=y$ for all $y\leq 0$.\\
	c) $h$ is a function but not bijective. It is a function because for every class in \textbf{U-M Courses}, there is a unique department(and cross-listed courses are considered distinctive). It is not bijective because it is onto but not 1-1. This is because each class has its own department, but two different classes within the same department will map to the same department.\\
	d) $k$ is a function but not bijective. It is a function because for every class in \textbf{U-M Courses}, there is a unique course number. It is not bijective because it is not onto or 1-1. This is because two different classes within different department can have the same course number, and some natural numbers do not get mapped as course number usually starts at 100.
\end{solution}


\subsection*{\probnum Fantastic Functions [18 points]}
For each of the functions below, determine whether it is (i) one-to-one, (ii) onto. Prove your answers.
\begin{qparts}
	\item $f \colon \mathbb{R} \to \mathbb R - \mathbb R^-,\ f(x) = e^{2x + 1}.$

	\item $g \colon \mathbb{R} - \left\{-\frac{2}{5}\right\} \to \mathbb{R} - \left\{\frac{3}{5}\right\},\ g(x) = \frac{3x - 1}{5x + 2}.$

	\item$h \colon \mathbb{Z} \times \mathbb {Z} \to \mathbb{Z},\ h(m, n) = |m|-|n|$
\end{qparts}


\begin{solution}
	a) $f$ is one-to-one but not onto\\
	% rewrite that into two formal proofs using tabular format
	\begin{tabular}{l l}
		i) Proof one-to-one:                                                      \\
		Assume $x_1,x_2\in\mathbb{R}$.                               & assumption \\
		$e^{2x_1+1}=e^{2x_2+1}$.                                     &            \\
		$2x_1+1=2x_2+1$.                                             &            \\
		$x_1=x_2$.                                                   &            \\
		Therefore, $f$ is one-to-one.                                &            \\
		ii) Proof: not onto                                                       \\
		Seeking contradiction, assume $y\in\mathbb{R}-\mathbb{R}^-$. & assumption \\
		$e^{2x+1}=y$.                                                &            \\
		$2x+1=\ln(y)$.                                               &            \\
		$x=\frac{\ln(y)-1}{2}$.                                      &            \\
		Since $\ln(y)$ is undefined for $y\le 0$, $x$ is undefined for $y\le 0$.  \\
		Therefore, $f$ is not onto.                                  &            \\
	\end{tabular}\\

	b) $g$ is one-to-one and onto.\\
	% rewrite that into two formal proofs using tabular format
	\begin{tabular}{l l}
		i) Proof one-to-one:                                                    \\
		Assume $x_1,x_2\in\mathbb{R}-\left\{-\frac{2}{5}\right\}$. & assumption \\
		$\frac{3x_1-1}{5x_1+2}=\frac{3x_2-1}{5x_2+2}$.             &            \\
		$3x_1(5x_2+2)-5x_1(3x_2-1)=3x_2(5x_1+2)-5x_2(3x_1-1)$.     &            \\
		$15x_1x_2+6x_1-15x_1x_2+5x_1=15x_2x_1+6x_2-15x_2x_1+5x_2$. &            \\
		$x_1=x_2$.                                                 &            \\
		Therefore, $g$ is one-to-one.                              &            \\
		ii) Proof onto:                                                         \\
		Assume $y\in\mathbb{R}-\left\{\frac{3}{5}\right\}$.        & assumption \\
		$\frac{3x-1}{5x+2}=y$.                                     &            \\
		$3x-1=y(5x+2)$.                                            &            \\
		$3x-1=5xy+2y$.                                             &            \\
		$3x-5xy=2y+1$.                                             &            \\
		$x=\frac{2y+1}{3-5y}$.                                     &            \\
		Therefore, $g$ is onto.                                    &            \\
	\end{tabular}\\
	c) $h$ is onto but not one-to-one.\\
	i) \\
	Proof not one-to-one:\\
	It is possible to have $h(m_1,n_1)=h(m_2,n_2)$ for $m_1\neq m_2$ and $n_1\neq n_2$. For example, $h(1,0)=h(0,1)=1$.\\
	Therefore, $h$ is not one-to-one.\\
	ii) \\
	Proof onto:\\
	To get negative values, set $m=0$. To get positive values, set $n=0$. To get 0, set $m=n$.\\
	All cases have been exhausted.\\
	Therefore, $h$ is onto.\\

\end{solution}


\subsection*{\probnum Comp$\circ$sition$($Functions$)$ [12 points]}
For each of the following pairs of functions $f$ and $g$, find $f \circ g$ and $g \circ f$. Make sure to include the domain and codomain of each composed function you give. If either can't be computed, explain why.
\begin{qparts}
	\item $f\colon \mathbb{N} \to \mathbb{Z}^{+},\; f(x) = x + 1$

	$g\colon\mathbb{Z}^{+}\to\mathbb{N},\; g(x) = x^2 - 1$

	\item $f\colon\mathbb{Z}\to\mathbb{R},\; f(x) = \left(\frac{3}{2} x + 3\right)^{3}$

	$g\colon\mathbb{R} \rightarrow \mathbb{R}_{\geq 0},\; g(x) = |x|$

	\textbf{Note:} $\mathbb{R}_{\geq 0}$ is the set of real numbers greater than or equal to 0.
\end{qparts}


\begin{solution}
	For composite functions to be computable, the domain of $f\circ g$ is the codomain of $g$ and the codomain of $f\circ g$ is the codomain of $f$.\\
	The domain of $g\circ f$ is the codomain of $f$ and the codomain of $g\circ f$ is the codomain of $g$.\\
	a) $f\circ g$, and $g\circ f$ are computable.\\
	$f\circ g\colon\mathbb{N}\to\mathbb{Z}^{+},\; f\circ g(x) = (x^2-1)+1=x^2$.\\
	$g\circ f\colon\mathbb{Z}^{+}\to\mathbb{N},\; g\circ f(x) = (x+1)^2-1=x^2+2x$.\\
	b) $f\circ g$ is not computable, but $g\circ f$ is computable.\\
	$f\circ g\colon\mathbb{R}\to\mathbb{R}_{\geq 0}$ is not computable because $f(x)$ is not defined for non-integers. i.e. when $g=1.2$.\\
	$g\circ f\colon\mathbb{R}\to\mathbb{R}_{\geq 0},\; g\circ f(x) = \left|\left(\frac{3}{2}x+3\right)^3\right|$.\\

\end{solution}
\end{document}
