\documentclass[12pt]{exam}

% essential packages
\usepackage{fullpage} % margin formatting
\usepackage{enumitem} % configure enumerate and itemize
\usepackage{amsmath, amsfonts, amssymb, mathtools} % math symbols
\usepackage{xcolor, colortbl} % colors, including in tables
\usepackage{makecell} % thicker \Xhline in table
\usepackage{graphicx} % images, resizing

\usepackage[dvipsnames]{xcolor}
\usepackage[framemethod=tikz]{mdframed}

\newcommand{\commentsection}[1]{
  \begin{mdframed}[roundcorner=10pt,leftmargin=1cm,%
                     rightmargin=1cm,backgroundcolor=SkyBlue!20,%
                     innertopmargin=\baselineskip,%
                     skipabove=\baselineskip,%
                     skipbelow=\baselineskip]
  #1
  \end{mdframed}
}

% sometimes needed packages
\usepackage{hyperref} % hyperlinks
% \hypersetup{colorlinks=true, urlcolor=blue}
% \usepackage{logicproof} % natural deduction
% \usepackage{tikz} % drawing graphs
% \usetikzlibrary{positioning}
% \usepackage{multicol}
% \usepackage{algpseudocode} % pseudocode

% paragraph formatting
\setlength{\parskip}{6pt}
\setlength{\parindent}{0cm}

% newline after Solution:
\renewcommand{\solutiontitle}{\noindent\textbf{Solution:}\par\noindent}


% less space before itemize/enumerate
\setlist{topsep=0pt}

% creates \filcl to grey out cells for groupwork grading
\newcommand{\filcl}{\cellcolor{gray!25}}

% creates \probnum to get the problem number
\newcounter{probnumcount}
\setcounter{probnumcount}{1}
\newcommand{\probnum}{\arabic{probnumcount}. \addtocounter{probnumcount}{1}}

% use roman numerals by default
\setlist[enumerate]{label={(\roman*)}}

% creates custom list environments for grading guidelines, question parts
\newlist{guidelines}{itemize}{1}
\setlist[guidelines]{label={}, left=0pt .. \parindent, nosep}
\newlist{gwguidelines}{enumerate}{1}
\setlist[gwguidelines]{label={(\roman*)}, nosep}
\newlist{qparts}{enumerate}{2}
\setlist[qparts]{label={(\alph*)}}
\newlist{qsubparts}{enumerate}{2}
\setlist[qsubparts]{label={(\roman*)}}
\newlist{stmts}{enumerate}{1}
\setlist[stmts]{label={(\roman*)}, nosep}
\newlist{pflist}{itemize}{4}
\setlist[pflist]{label={$\bullet$}, nosep}
\newlist{enumpflist}{enumerate}{4}
\setlist[enumpflist]{label={(\arabic*)}, nosep}

\printanswers

\newcommand{\prevhwnum}{4}
\newcommand{\hwnum}{4}

\begin{document}
\pagebreak
\section*{Grading of Groupwork \prevhwnum{}}
Using the solutions and Grading Guidelines, grade your Groupwork \prevhwnum{} Problems:
\begin{itemize}
	\item Use the table below to grade your past groupwork submission and calculate scores.
	\item While grading, mark up your past submission. Include this with the table when you submit your grading.
	\item Write whether your submission achieved each rubric item. If it didn't achieve one, say why not.
	\item For extra credit, write positive comment(s) about your work.
	\item You don't have to redo problems correctly, but it is recommended!
	\item See ``All About Groupwork" on Canvas for more detailed guidance, and what to do if you change groups.
\end{itemize}

\begin{center}
	\resizebox{\textwidth}{!}{\begin{tabular}{| c | c | c | c | c | c | c | c | c | c | c | c | c |}
			\hline
			          & (i)    & (ii)   & (iii)  & (iv)   & (v)    & (vi)   & (vii)  & (viii) & (ix)   & (x)    & (xi)   & Total:           \\
			\hline
			Problem 1 & 0      & 0      & 0      & \filcl & \filcl & \filcl & \filcl & \filcl & \filcl & \filcl & \filcl & \hspace{1cm}0/12 \\
			\hline
			Problem 2 & 1      & 1      & 0      & 1      & 0      & 0      & 1      & 0      & \filcl & \filcl & \filcl & \hspace{1cm}5/8  \\
			\Xhline{1.25pt}
			Total:    & \filcl & \filcl & \filcl & \filcl & \filcl & \filcl & \filcl & \filcl & \filcl & \filcl & \filcl & \hspace{1cm}5/20 \\
			\hline
		\end{tabular}}
\end{center}

\section*{Comments}


\commentsection{I went to OH and tried. Good effort although the solutions were different from the answer key.}
\setcounter{probnumcount}{1}
\section*{Groupwork \hwnum{} Problems}


\subsection*{\probnum Mostly Rational [12 points]}
Show that if $r$ is an irrational number, there is a unique integer $n$ such that the distance between $r$ and $n$ is \textit{strictly} less than $\frac 12.$

\begin{solution}
	\textcolor{red}{Different from the answer key, but still correct?}\\
	Proof by contradiction:\\
	\begin{tabular}{ll}
		Assume $r\notin \mathbb{Q}, \exists n_1\exists n_2 [|r-n_1|<\frac {1}{2} \land |r-n_2|<\frac {1}{2}](n_1,n_2 \in \mathbb{Z})$ & premise             \\
		$|n_1-n_2|\le |r-n_1|+|r-n_2|$                                                                                                & triangle inequality \\
		$|n_1-n_2|\le \frac{1}{2} + \frac{1}{2} = 1$                                                                                  & substitution        \\
		Since $n_1,n_2\in \mathbb{Z}$,                                                                                                                      \\
		they cannot have a distance $< 1$,                                                                                                                  \\
		this contradicts with the premise                                                                                             & contradiction       \\
	\end{tabular}
	\\Therefore, the original proposition holds by proof by contradiction.
\end{solution}


\subsection*{\probnum Set in Stone [8 points]}
Prove using set identities that
$$(A \cap C) - (B \cap A) = (C - B) \cap A$$
for any three sets $A,\ B$ and $C.$

\begin{solution}
	\textcolor{red}{Different from the answer key, but still correct?}\\
	\begin{tabular}{ll}
		Let $x\in (A \cap C) - (B \cap A)$                                                 & premise                         \\
		$x\in (A \cap C) \land x\notin (B \cap A)$                                         & definition of minus             \\
		$x\in A \land x\in C \land (x\notin B \lor x\notin A)$                             & DeMorgan's Laws                 \\
		$(x\in A \land x\in C \land x\notin B) \lor (x\in A \land x\in C \land x\notin A)$ & distributive                    \\
		$(x\in A \land x\in C \land x\notin B) \lor F$                                     &                                 \\
		$x\in (C-B)\cap A$                                                                 & definition of minus / intersect \\

		\\Conversely,\\
		Let $x\in (C-B)\cap A$                                                             & premise                         \\
		$(x\in C \land x\notin B)\cap A$                                                   & definition of minus             \\
		$(x\in C \cap x \in A) \land (x\notin B \cap A) $                                  & distributive                    \\
		$x\in (A \cap C) \land x\notin (B \cap A)$                                         & definition of minus             \\
		$x\in (A \cap C) - (B \cap A)$                                                                                       \\


		\\Thus, proved that each side is a subset of the other.
	\end{tabular}
\end{solution}
\end{document}
