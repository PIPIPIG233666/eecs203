\documentclass[12pt]{exam}

% essential packages
\usepackage{fullpage} % margin formatting
\usepackage{enumitem} % configure enumerate and itemize
\usepackage{amsmath, amsfonts, amssymb, mathtools} % math symbols
\usepackage{xcolor, colortbl} % colors, including in tables
\usepackage{makecell} % thicker \Xhline in table
\usepackage{graphicx} % images, resizing

% sometimes needed packages
\usepackage{hyperref} % hyperlinks
\usepackage{tikz} % drawing graphs
\usetikzlibrary{positioning}

\usepackage[dvipsnames]{xcolor}
\usepackage[framemethod=tikz]{mdframed}

\newcommand{\commentsection}[1]{
  \begin{mdframed}[roundcorner=10pt,leftmargin=1cm,%
                     rightmargin=1cm,backgroundcolor=SkyBlue!20,%
                     innertopmargin=\baselineskip,%
                     skipabove=\baselineskip,%
                     skipbelow=\baselineskip]
  #1
  \end{mdframed}
}

% paragraph formatting
\setlength{\parskip}{6pt}
\setlength{\parindent}{0cm}

% newline after Solution:
\renewcommand{\solutiontitle}{\noindent\textbf{Solution:}\par\noindent}

% less space before itemize/enumerate
\setlist{topsep=0pt}

% creates \filcl to grey out cells for groupwork grading
\newcommand{\filcl}{\cellcolor{gray!25}}

% creates \probnum to get the problem number
\newcounter{probnumcount}
\setcounter{probnumcount}{1}
\newcommand{\probnum}{\arabic{probnumcount}. \addtocounter{probnumcount}{1}}

% use roman numerals by default
\setlist[enumerate]{label={(\roman*)}}

% creates custom list environments for grading guidelines, question parts
\newlist{guidelines}{itemize}{1}
\setlist[guidelines]{label={}, left=0pt .. \parindent, nosep}
\newlist{gwguidelines}{enumerate}{1}
\setlist[gwguidelines]{label={(\roman*)}, nosep}
\newlist{qparts}{enumerate}{2}
\setlist[qparts]{label={(\alph*)}}
\newlist{qsubparts}{enumerate}{2}
\setlist[qsubparts]{label={(\roman*)}}
\newlist{stmts}{enumerate}{1}
\setlist[stmts]{label={(\roman*)}, nosep}
\newlist{pflist}{itemize}{4}
\setlist[pflist]{label={$\bullet$}, nosep}
\newlist{enumpflist}{enumerate}{4}
\setlist[enumpflist]{label={(\arabic*)}, nosep}

\printanswers

\newcommand{\prevhwnum}{8}
\newcommand{\hwnum}{8}

\begin{document}
\pagebreak
\section*{Grading of Groupwork \prevhwnum{}}
Using the solutions and Grading Guidelines, grade your Groupwork \prevhwnum{} Problems:
\begin{itemize}
	\item Use the table below to grade your past groupwork submission and calculate scores.
	\item While grading, mark up your past submission. Include this with the table when you submit your grading.
	\item Write whether your submission achieved each rubric item. If it didn't achieve one, say why not.
	\item For extra credit, write positive comment(s) about your work.
	\item You don't have to redo problems correctly, but it is recommended!
	\item See ``All About Groupwork" on Canvas for more detailed guidance, and what to do if you change groups.
\end{itemize}

\begin{center}
	\resizebox{\textwidth}{!}{\begin{tabular}{| c | c | c | c | c | c | c | c | c | c | c | c | c |}
			\hline
			          & (i)    & (ii)   & (iii)  & (iv)   & (v)    & (vi)   & (vii)  & (viii) & (ix)   & (x)    & (xi)   & Total:           \\
			\hline
			Problem 1 & 0      & 0      & 0      & 0      & 0      & 0      & 0      & \filcl & \filcl & \filcl & \filcl & \hspace{1cm}0/15 \\
			\hline
			Problem 2 & 0      & 0      & 0      & 0      & \filcl & \filcl & \filcl & \filcl & \filcl & \filcl & \filcl & \hspace{1cm}0/15 \\
			\Xhline{1.25pt}
			Total:    & \filcl & \filcl & \filcl & \filcl & \filcl & \filcl & \filcl & \filcl & \filcl & \filcl & \filcl & \hspace{1cm}0/30 \\
			\hline
		\end{tabular}}
\end{center}

\section*{Comments}
\commentsection{I had no clue where to start with counting but I tried my best. I think I need to review the counting rules.}

\setcounter{probnumcount}{1}
\section*{Groupwork \hwnum{} Problems}

\subsection*{\probnum Commit Tea Party [15 points]}
Two committees are having a meeting. If there are 12 people in each committee, how many different ways can they sit around a table given the following restrictions? Note that two orderings are considered equal if each person has the same two neighbors (without distinguishing their left and right neighbors).

\begin{qparts}
	\item There are no restrictions on seating.
	\item Two people in the same committee cannot be neighbors.
	\item Everybody must have exactly two neighbors from their committee.
	\item Everybody must have exactly one neighbor from their committee.
\end{qparts}

\begin{solution}
	There are $24$ people in total and $2$ committees.\\
	a) No restrictions on seating.\\
	Therefore, it is a combination problem.\\
	$C(24, 12) = \frac{24!}{12!12!} = 2704156$ ways.\\
	\textcolor{red}{$\frac{24!}{2\cdot 24}.$}\\
	b) Two people in the same committee cannot be neighbors.\\
	Therefore, it is a permutation problem.\\
	$P(24,2) = 24\cdot 23 = 552$ ways.\\
	\textcolor{red}{$\frac{11! \cdot 12!}{2}$}\\
	c) Everybody must have exactly two neighbors from their committee.\\
	Therefore, it is a permutation problem.\\
	$P(24,3) = 24\cdot 23\cdot 22 = 12144$ ways.\\
	\textcolor{red}{0, impossible}\\
	d) Everybody must have exactly one neighbor from their committee.\\
	Therefore, it is a permutation problem.\\
	$P(24,2) = 24\cdot 23 = 552$ ways.\\
	\textcolor{red}{$\frac{12! \cdot 12!}{2\cdot 6}$}\\
\end{solution}

\subsection*{\probnum Hiking Extravaganza [15 points]}
Prove that every complete $n$-node weighted graph (with all possible edges) with $n \ge 1$ and all distinct edge weights has a (possibly non-simple) path of $n-1$ edges along which the edge weights are strictly increasing.

\textbf{Hint:} Start by placing a hiker on each node. Try to show that the hikers can walk paths of \emph{total} length $n(n-1)$, each along increasing-weight paths.

\begin{solution}
	\textcolor{red}{correct answer}\\
	Place a hiker at each node of $G$. Consider the edges of $G$ in order of increasing weight. When each edge $\{u, v\}$ is considered, have the hiker currently at $u$ walk across the edge to $v$, and the hiker currently at $v$ walk across the edge to $u$ (so they switch places). At the end of this process, each edge will have traversed by exactly 2 hikers (one in each direction).

	Because our process visits the edges in order of increasing weight, the path walked by each hiker is guaranteed to be an increasing-weight path.

	Next we want to show that at least one hiker's path contained (at least) $n-1$ edges. To do this, we'll first consider average number of edges in a hiker's path. \\
	The sum of the edges traversed by all hikers is twice the number of edges in the graph, since each edge was traversed by 2 hikers (one in each direction).\\
	The total number of edges can be found in a variety of ways, including via the Handshake Theorem, which we do here. The Handshake theorem tells us that the sum of the degrees in a graph is twice the number of edges. For a complete graph on $n$ nodes, each node has degree $n-1$. The sum of degrees is $n(n-1)$, so the total number of edges is $\frac{n(n-1)}{2}.$\\
	So the sum of the edges traversed by all hikers is $2\cdot \frac{n(n-1)}{2} = n(n-1).$\\
	The average number of edges in a hiker's path is then $\frac{n(n-1)}{n}=n-1.$\\

	Since the average edges-in-path is $n-1$, then at least one hiker must have walked a path with $\ge n-1$ edges. (Because that's how averages work.)


\end{solution}

\end{document}
