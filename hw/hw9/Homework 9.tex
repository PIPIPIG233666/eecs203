\documentclass[12pt]{exam}

% essential packages
\usepackage{fullpage} % margin formatting
\usepackage{enumitem} % configure enumerate and itemize
\usepackage{amsmath, amsfonts, amssymb, mathtools} % math symbols
\usepackage{xcolor, colortbl} % colors, including in tables
\usepackage{makecell} % thicker \Xhline in table
\usepackage{graphicx} % images, resizing

% sometimes needed packages
\usepackage{hyperref} % hyperlinks
% \hypersetup{colorlinks=true, urlcolor=blue}
% \usepackage{logicproof} % natural deduction
% \usepackage{tikz} % drawing graphs
% \usetikzlibrary{positioning}
% \usepackage{multicol}
% \usepackage{algpseudocode} % pseudocode

% paragraph formatting
\setlength{\parskip}{6pt}
\setlength{\parindent}{0cm}

% newline after Solution:
\renewcommand{\solutiontitle}{\noindent\textbf{Solution:}\par\noindent}

% less space before itemize/enumerate
\setlist{topsep=0pt}

% creates \filcl to grey out cells for groupwork grading
\newcommand{\filcl}{\cellcolor{gray!25}}

% creates \probnum to get the problem number
\newcounter{probnumcount}
\setcounter{probnumcount}{1}
\newcommand{\probnum}{\arabic{probnumcount}. \addtocounter{probnumcount}{1}}

% use roman numerals by default
\setlist[enumerate]{label={(\roman*)}}

% creates custom list environments for grading guidelines, question parts
\newlist{guidelines}{itemize}{1}
\setlist[guidelines]{label={}, left=0pt .. \parindent, nosep}
\newlist{gwguidelines}{enumerate}{1}
\setlist[gwguidelines]{label={(\roman*)}, nosep}
\newlist{qparts}{enumerate}{2}
\setlist[qparts]{label={(\alph*)}}
\newlist{qsubparts}{enumerate}{2}
\setlist[qsubparts]{label={(\roman*)}}
\newlist{stmts}{enumerate}{1}
\setlist[stmts]{label={(\roman*)}, nosep}
\newlist{pflist}{itemize}{4}
\setlist[pflist]{label={$\bullet$}, nosep}
\newlist{enumpflist}{enumerate}{4}
\setlist[enumpflist]{label={(\arabic*)}, nosep}

\printanswers

\newcommand{\prevhwnum}{8}
\newcommand{\hwnum}{9}

\begin{document}
%%%%%%%%%%%%%%% TITLE PAGE %%%%%%%%%%%%%%%
\section*{Individual Portion}

\subsection*{\probnum How Many pa55words? [12 points]}
A password consists of exactly 6 characters, where each character is either a lowercase letter (a-z) or a digit (0-9). However, the password must contain at least 2 digits and at least 2 lowercase letters. How many different passwords are possible?

\begin{solution}
	The total number of passwords is the total number of passwords minus the number of passwords that do not satisfy the conditions.\\
	The total number of passwords is $36^6$ since there are 26 lowercase letters and 10 digits to choose from for each of the 6 characters.\\
	Consider the following cases:
	\begin{itemize}
		\item Case 1: $<$ 2 digits\\
		      subcase 1: 0 digits\\
		      In the case where there are no digits, there must be 6 lowercase letters.\\
		      $_{6}C_{6} \cdot 26^6$\\
		      subcase 2: 1 digit\\
		      In the case where there is 1 digit, there must be 5 lowercase letters.\\
		      $_{6}C_{1} \cdot 26^5 \cdot 10$\\
		\item Case 2: $<$ 2 lowercase letters\\
		      subcase 1: 0 lowercase letters\\
		      In the case where there are no lowercase letters, there must be 6 digits.\\
		      $_{6}C_{6} \cdot 10^6$\\
		      subcase 2: 1 lowercase letter\\
		      In the case where there is 1 lowercase letter, there must be 5 digits.\\
		      $_{6}C_{1} \cdot 10^5 \cdot 26$\\
	\end{itemize}
	Therefore, the answer is $36^6 - (_{6}C_{6} \cdot 26^6 + _{6}C_{1} \cdot 26^5 \cdot 10 + _{6}C_{6} \cdot 10^6 + _{6}C_{1} \cdot 10^5 \cdot 26)$.

\end{solution}


\subsection*{\probnum (Not So) Round and Round [12 points]}
Suppose we have a square-shaped table which seats 3 people on each side. How many ways are there to seat 12 people at the table where seatings are considered the same if everyone is in the same group of 3 on a side?

\begin{solution}
	We can start by overcounting the number of ways to seat 12 people at the table.\\
	There are $12!$ ways to seat 12 people at the table without any restrictions.\\
	However, seatings are considered the same if everyone is in the same group of 3 on a side.\\
	Consider 4 groups of 3 people.\\
	There are $4!$ ways to arrange the groups.\\
	There are $3!$ ways to arrange the people within 4 groups.\\
	Therefore, the answer is $\frac{12!}{4! \cdot 3! \cdot 3! \cdot 3! \cdot 3!}$.

\end{solution}


\subsection*{\probnum My backpack is too heavy [12 points]}
How many ways are there to distribute seven textbooks into eleven backpacks if \textbf{each backpack must have at most one textbook in it} and
\begin{qparts}
	\item all of the textbooks and all of the backpacks are unique?
	\item all of the textbooks are unique, but all of the backpacks are identical?
	\item all of the textbooks are identical, but the backpacks are all unique?
	\item all of the textbooks and all of the backpacks are identical?
\end{qparts}

\begin{solution}
	Unique means order matters. Identical means order does not matter.\\
	\begin{qparts}
		\item Since all of the textbooks and all of the backpacks are unique, order matters.\\
		Therefore, it is a permutation problem.\\
		% 11 P 7
		The answer is $_{11}P_{7}$.
		\item Since all of the backpacks are identical, no matter how the textbooks are distributed, the backpacks are the same.\\
		Therefore, there is only one way to distribute the textbooks.\\
		The answer is 1.
		\item Order does not matter for the textbooks.\\
		Therefore, it is a combination problem.\\
		% 11 C 7
		The answer is $_{11}C_{7}$.
		\item Since all of the textbooks and all of the backpacks are identical, no matter how the textbooks are distributed, the backpacks are the same.\\
		Therefore, there is only one way to distribute the textbooks.\\
		The answer is 1.
	\end{qparts}
\end{solution}


\subsection*{\probnum Seeressously? [12 points]}
How many strings with six or more characters can be formed from the letters in ``SEERESS"? \textbf{Simplify your answer.} You may use a calculator to help you simplify.

\begin{solution}
	We want to form 6 or more characters from the 7 letters in ``SEERESS".\\
	The maximum number of characters we can form is 7.\\
	Therefore, we can form 6, 7 characters with the 3Es, 3Ss, and 1R.\\
	Consider the following cases:
	\begin{itemize}
		\item Case 1: 7 characters\\
		      We use all 7 letters.\\
		      We start by overcounting the number of ways to form 7 characters.\\
		      There are $7!$ ways to form 7 characters.\\
		      However, the 3Es and 3Ss are identical.\\
		      There are $3!$ ways to arrange the 3Es and $3!$ ways to arrange the 3Ss.\\
		      Therefore, the answer is $\frac{7!}{3! \cdot 3!\cdot 1!} = \frac{7!}{3! \cdot 3!}= \frac{7\cdot 6\cdot 5\cdot 4}{3\cdot 2\cdot 1} = 140$.\\\\
		\item Case 2: 6 characters\\
		      We use 6 letters.\\
		      There are $6!$ ways to form 6 characters.\\
		      subcase 1: 3Es, 2Ss, 1R\\
		      The 3Es are identical.\\
		      There are $3!$ ways to arrange the 3Es.\\
		      Therefore, the answer is $\frac{6!}{3! \cdot 2!\cdot 1!} = 60$.\\
		      subcase 2: 3Es, 3Ss\\
		      The 3Es and 3Ss are identical.\\
		      There are $3!$ ways to arrange the 3Es and $3!$ ways to arrange the 3Ss.\\
		      Therefore, the answer is $\frac{6!}{3! \cdot 3!\cdot 1!} = 20$.\\
		      subcase 3: 3Ss, 2Es, 1R\\
		      The 3Ss are identical.\\
		      There are $3!$ ways to arrange the 3Ss.\\
		      Therefore, the answer is $\frac{6!}{3! \cdot 2!\cdot 1!} = 60$.\\
	\end{itemize}

\end{solution}


\subsection*{\probnum Jackpot! [12 points]}
Suppose there is a lottery where the organizers pick a set of 11 distinct numbers. A player then picks 7 distinct numbers and wins when all 7 are in the set chosen by the organizers. Numbers chosen by both the players and organizers come from the set $\{ 1, 2, ..., 80 \}$.

\begin{qparts}
	\item Let the sample space, $S$, be all the sets of 7 numbers the player can choose. What is $|S|$?
	\item Let $E$ be the event that all the numbers the player chooses are in the winning set. What is $|E|$?
	\item What is the probability of winning? As a reminder, you may leave your answer unsimplified.
	\item Alternatively we can choose a different sample space $S'$ which is all the sets of 11 numbers that the organizers can choose. What is $|S'|$? What is the event $E'$ that the player has a winning set, and what is $|E'|$?
	\item What is the probability of winning given your answer to (d)? Use a calculator to verify this is the same as your answer to (c).
\end{qparts}

\begin{solution}
	\begin{qparts}
		\item There are 80 numbers to choose from.\\
		Therefore, the answer is $_{80}C_{7}$.
		\item There are 11 numbers to choose from.\\
		Therefore, the answer is $_{11}C_{7}$.
		\item The probability of winning is $\frac{|E|}{|S|} = \frac{_{11}C_{7}}{_{80}C_{7}}$.
		\item There are 80 numbers to choose from.\\
		Therefore, the answer is $_{80}C_{11}$.\\
		The event $E'$ is the event that all the numbers the player chooses are in the winning set.\\
		Since 7 of the 11 numbers are chosen by the player, the remaining 4 numbers must be chosen by the organizers.\\
		The 4 numbers chosen by the organizers must be from the 73 numbers not chosen by the player.\\
		Therefore, the answer is $_{80-7}C_{11-7} = _{73}C_{4}$.\\
		\item The probability of winning is $\frac{|E'|}{|S'|} = \frac{_{73}C_{4}}{_{80}C_{11}}$.
		This is the same as the probability of winning in part (c).
	\end{qparts}
\end{solution}


\subsection*{\probnum Rollin' in the Deep [12 points]}
You roll three, fair, 6-sided dice. What is the probability that the sum of the dice is less than 17?

\begin{solution}
	The probability that the sum of the dice is less than 17 is 1 minus the probability that the sum of the dice is 17 or more.\\
	There are $6^3$ total outcomes when rolling 3 dice.\\
	Consider the following cases:
	\begin{itemize}
		\item Case 1: The sum of the dice is 17.\\
		      There are 3 ways to get a sum of 17: 6, 6, 5; 6, 5, 6; 5, 6, 6.\\
		\item Case 2: The sum of the dice is 18.\\
		      There is only 1 way to get a sum of 18: 6, 6, 6.\\
	\end{itemize}
	Therefore, there are 4 ways to get a sum of 17 or more.\\
	Thus, the probability that the sum of the dice is less than 17 is $1 - \frac{4}{6^3} = \frac{212}{216}$.


\end{solution}


\subsection*{\probnum Addition Condition [12 points]}
A pair of six-sided dice is rolled, but you do not see the results. You ask your friend whether at least one die came up six, and they respond yes.

What is the probability that the sum of the numbers that came up on the two dice is seven, given the information provided by your friend?

\begin{solution}
	Let $A$ be the event that the sum of the numbers that came up on the two dice is seven.\\
	Let $B$ be the event that at least one die came up six.\\
	The probability that the sum of the numbers that came up on the two dice is seven given the information provided by your friend is $\frac{P(A \cap B)}{P(B)}$.\\
	The sum of the numbers that came up on the two dice is seven and at least one die came up six.\\
	There are 2 ways to get a sum of 7: 1, 6; 6, 1.\\
	There are 11 ways to get at least one die to come up six. 6 ways to get a 6 on the first die and 5 ways to get a number other than 6 on the second die. 5 ways to get a number other than 6 on the first die and 6 ways to get a 6 on the second die:\\
	$(6, 1), (6, 2), (6, 3), (6, 4), (6, 5), (6, 6), (1, 6), (2, 6), (3, 6), (4, 6), (5, 6)$.\\
	Therefore, the answer is $\frac{2}{11}$.

\end{solution}


\subsection*{\probnum Conditional probability \& Independence [16 points]}

Suppose you put two dice in a bag: one of the dice has a 6 on every face, and the other is a standard 6-sided die. You choose one die at random, roll it, and get a 6.

\begin{qparts}
	\item If you roll the same die, what is the probability that the next roll is also a 6?
	\item If you roll the same die both times, are rolling a 6 on the first roll and rolling a 6 on the second roll independent events?
\end{qparts}

\begin{solution}
	The probability of getting a 6 with a standard 6-sided die is $\frac{1}{6}$.\\
	The probability of getting a 6 with a die with a 6 on every face is 1.\\
	\begin{qparts}
		\item You either choose the die with a 6 on every face or the standard 6-sided die.\\
		The probability of choosing the die with a 6 on every face is $\frac{1}{2}$.\\
		The probability of choosing the standard 6-sided die is $\frac{1}{2}$.\\
		Therefore, the probability of getting a 6 is either $\frac{1}{2}\cdot 1 + \frac{1}{2}\cdot \frac{1}{6} = \frac{7}{12}$.\\

		\item Let $E$ be the event that you roll a 6 on the first roll and $F$ be the event that you roll a 6 on the second roll.\\
		$P(F) = \frac{1}{2}\cdot 1 + \frac{1}{2}\cdot \frac{1}{6} = \frac{7}{12}$.\\
		$P(F \mid E) = \frac{1}{2}\cdot \frac{1}{6} \cdot \frac{1}{6} + \frac{1}{2}\cdot 1 \cdot 1$\\
		They are not equal, so the events are not independent.
	\end{qparts}

\end{solution}
\end{document}
